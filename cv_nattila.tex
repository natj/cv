%\documentclass[a4paper, onecolumn, 11pt]{article}
\documentclass[letterpaper, onecolumn, 11pt]{article}

\usepackage[a4paper,left=2cm,top=2.0cm,bottom=1.5cm,right=2cm, headheight=0.1cm, footskip=0.5cm]{geometry}

\usepackage[utf8]{inputenc}
\usepackage{graphicx} 		% Add graphics capabilities
\usepackage{color}
\usepackage{array}
\usepackage{xcolor}
\usepackage[colorlinks = true,
            linkcolor = blue,
            urlcolor  = blue,
            citecolor = blue,
            anchorcolor = blue]{hyperref}

\usepackage{enumitem}
\setitemize{noitemsep,topsep=0pt,parsep=0pt,partopsep=0pt,leftmargin=*}

\usepackage{amsmath}  		% Better maths support
\usepackage{amssymb}

\usepackage{titlesec}
%\titleformat*{\title}{\bfseries}
\titleformat*{\section}{\bfseries}
%\titleformat*{\subsection}{\itshape}
%\titleformat*{\subsubsection}{\normalsize}
%\titlespacing\section{0pt}{12pt plus 4pt minus 2pt}{0pt plus 2pt minus 2pt}
%\titlespacing\subsection{0pt}{12pt plus 4pt minus 2pt}{0pt plus 2pt minus 2pt}
%\titlespacing\subsubsection{0pt}{12pt plus 4pt minus 2pt}{0pt plus 2pt minus 2pt}

\usepackage{fancyhdr}
\pagestyle{fancy}
\renewcommand{\headrulewidth}{0pt} % Remove line at top

\lhead{\textcolor{darkgray}{\emph{N\"attil\"a}}}
\chead{\textcolor{darkgray}{}}
\rhead{\textcolor{darkgray}{\emph{CV}}}
%\rhead{\textcolor{darkgray}{\emph{Publication list}}}
\cfoot{\textcolor{darkgray}{\thepage}/3} %\pageref{LastPage}}



\newcommand\blfootnote[1]{%
  \begingroup
  \renewcommand\thefootnote{}\footnote{#1}%
  \addtocounter{footnote}{-1}%
  \endgroup
}


%\Huge Curriculum Vitae}
%\author{joonas.nattila@su.se}

%\author{jnattila@flatironinstitute.org}
%\date{}
 

\newcolumntype{L}{>{\raggedleft}p{0.11\textwidth}}
\newcolumntype{R}{p{0.89\textwidth}}
\definecolor{lightgray}{gray}{0.8}
\newcommand\VRule{\color{lightgray}\vrule width 0.5pt}

%%%%%%%%%
\usepackage{etaremune}
\makeatletter
\long\def\thebibliography#1{%
  \section*{\refname}%
  \@mkboth{\MakeUppercase\refname}{\MakeUppercase\refname}
  \settowidth{\dimen0}{\@biblabel{#1}}%
  \setlength{\dimen2}{\dimen0}%
  \addtolength{\dimen2}{\labelsep}
  \sloppy
  \clubpenalty 4000 
  \@clubpenalty 
  \clubpenalty 
  \widowpenalty 4000
  \sfcode `\.\@m
  \renewcommand{\labelenumi}{\@biblabel{\theenumi}} % labels like [3], [2], [1]
  \begin{etaremune}[labelwidth=\dimen0,leftmargin=\dimen2]\@openbib@code
}
\def\endthebibliography{\end{etaremune}}
\def\@bibitem#1{%
  \item \if@filesw\immediate\write\@auxout{\string\bibcite{#1}{\the\value{enumi}}}\fi\ignorespaces
}
\makeatother
%%%%%%%%%%%%%%%%%%%%%

 
\begin{document}
%\setlength{\extrarowheight}{0.5ex}


%\pagenumbering{gobble}
%\maketitle

\begin{center}
\textbf{CURRICULUM VITAE}
\end{center}


\section*{PERSONAL INFORMATION}
\vspace{-0.3cm}
\begin{minipage}[ht]{0.6\textwidth}
Last name, First name: N\"attil\"a, Joonas\\
ORCID: \href{https://orcid.org/0000-0002-3226-4575}{\nolinkurl{0000-0002-3226-4575}} \\
%Google Scholar: \href{https://scholar.google.com/citations?user=Wq1ZTb8AAAAJ}{Wq1ZTb8AAAAJ}\\
%Date of birth: June 25, 1989\\
%Nationality: Finnish\\
%Languages: English, Finnish (native), Swedish  \\
webpage: \url{http://natj.github.io} \\
\end{minipage}
%
\begin{minipage}[ht]{0.40\textwidth}
    Center for Computational Astrophysics \\
    Flatiron Institute \\
    162 5th Avenue, New York, NY-10010 \\
%    webpage: \url{http://natj.github.io} \\
%    CV updated: \today\\
\end{minipage}


\vspace{-0.5cm}
\section*{EDUCATION} %\blfootnote{\today}
\vspace{-0.3cm}
\begin{tabular}{L!{\VRule}R}
2014--2017 & PhD (with honors; ranked in the top 10\% of dissertations internationally)\\
           & Faculty of Mathematics and Natural Sciences, University of Turku, Finland\\[0.5ex]
%         & Advisor: Juri Poutanen\\[0.5ex]
%  & \footnotesize{Title: X-ray bursts as a tool to constrain the equation of state of the ultra-dense matter inside neutron stars} \\[1ex]
%  & \small{weighted GPA of $4.48/5$} / 4.5/10 GPA\\[1ex]  
2012--2013 & MSc, Degree Program in Physics\\
         & Faculty of Science, University of Oulu, Finland\\[0.5ex]
    % \scriptsize{($4.71/5$ GPA)}\\ & \small{weighted GPA of $4.75/5$} \\[1ex]  
%2008--2012& BSc, Degree Program in Physics\\
%         & Faculty of Science, University of Oulu, Finland\\
  %\scriptsize{($4.51/5$ GPA)}\\
%  & \small{weighted GPA of $4.48/5$} \\[1ex]   
\end{tabular}

\vspace{-0.3cm}
\section*{CURRENT POSITIONS}
\vspace{-0.3cm}
\begin{tabular}{L!{\VRule}R}
 2023--\phantom{3000} & Associate Research Scientist\\
                      & Columbia Astrophysics Laboratory, Columbia University, New York, USA\\[0.5ex]
 2019--2023 & Postdoctoral Researcher (Joint Columbia/Flatiron Institute Research Fellow)\\
            & Department of Physics, Columbia University, New York, USA\\
            & Center for Computational Astrophysics, Flatiron Institute, New York, USA\\
%            & Mentors: Andrei Beloborodov, Lorenzo Sironi (CU); James Cho, Sasha Philippov (FI)\\
\end{tabular}


\vspace{-0.3cm}
\section*{PREVIOUS POSITIONS}
\vspace{-0.3cm}
\begin{tabular}{L!{\VRule}R}
2018--2019 & Postdoctoral Researcher (Nordita Fellow) \\
           & Nordic Institute for Theoretical Physics (Nordita), Stockholm, Sweden \\[0.5ex]
%           & Mentor: Axel Brandenburg\\[0.5ex]
2014--2017 & PhD Student \\
           & Faculty of Mathematics and Natural Sciences, University of Turku, Finland\\[0.5ex]
2011--2013 & Research Assistant\\
           & Astronomy Department, University of Oulu, Finland
\end{tabular}


\vspace{-0.3cm}
\section*{FELLOWSHIPS, AWARDS, AND RECOGNITION}
\vspace{-0.3cm}
\begin{tabular}{L!{\VRule}R}
    2022 & Ref. \cite{2022PhRvL.128g5101N} on cover of \textit{PRL} (vol. 128, issue 7), \textit{Editor's Suggestion}, and \href{https://physics.aps.org/articles/v15/20}{APS \textit{Viewpoint} } \\
  2021 & Mikael Bj\"ornberg Prize for Young Theoretical Physicist; $10\,000$ EUR, Finland \\
  2021--2023 & Flatiron Research Fellow, Center for Computational Astrophysics, USA\\
  2019 & Joint Princeton University/Flatiron Postdoctoral Research Fellowship (\textit{Declined}) \\
  2018--2019 & Nordita Fellow, Nordic Institute for Theoretical Physics, Sweden\\
  2018 & V\"ais\"al\"a Prize 2018: Outstanding Thesis in Astronomy, Finland \\
  2018 & Turku Finnish University Society Prize for Best Doctoral Dissertation, Finland \\
  2018 & PCS Best Doctoral Thesis of 2017 Prize, Finland \\
  2016 & Nordita Visiting PhD Fellow, Nordita, Sweden \\
  2015--2017 & University of Turku Chemical and Physical Sciences Program PhD Scholar Awardee\\
  2014--2015 & V\"ais\"al\"a Foundation PhD Scholar Awardee \\
\end{tabular}


%\vspace{-0.3cm}
\section*{RESEARCH INTERESTS}
\vspace{-0.3cm}
I am a computational astrophysicist studying the dynamics of astrophysical fluids and plasmas. 
I have a broad range of research interests, including:
\\[0.9ex]
\noindent \textbf{High-energy astrophysics}:
accretion flows around black holes; %Energization of accretion flows around black holes;
thermonuclear X-ray bursts; %fluid dynamics of thermonuclear X-ray bursts;
magnetar flares, fast radio bursts; %radiative plasma physics of magnetar giant bursts and FRBs;
neutron star mergers; %electromagnetic precursors of neutron star mergers;
pulsar radio emission %magnetospheres and radio emission

\noindent \textbf{Plasma physics}: 
turbulence; %in magnetically-dominated plasmas; 
collisionless shocks;
magnetic reconnection

\noindent \textbf{Fluid dynamics}: 
storm dynamics in hot-exoplanet atmospheres

\noindent \textbf{Nuclear physics}: 
equation of state of ultra-dense matter inside neutron stars

\noindent \textbf{Computer sciences}: 
high-performance computing; machine learning; Monte Carlo methods

\noindent \textbf{Mathematics}: 
cellular automata models

%\vspace{-0.3cm}
\section*{MENTORING AND SUPERVISION}
\vspace{-0.3cm}
\noindent
In total, advised/co-advised 1 PhD, 3 MSc, and 2 BSc thesis; advising 2 PreDoc projects (5-month PhD internships at Flatiron Institute).\\[1.0ex]
\begin{tabular}{L!{\VRule}R}
    2023       & PreDoc advisor, Flatiron Pre-Doctoral Program (PhD student T. Ha), USA\\
          2023 & PreDoc advisor, Flatiron Pre-Doctoral Program (PhD student V. Loktev), USA\\
    2017--2020 & PhD co-advisor (T. Salmi), University of Turku, Finland\\
    2019--2020 & MSc advisor (J. Hope), Nordita/University of Bath, Sweden/UK\\
%    2019--2020 & Host for Nordita Visiting PhD students (M. Bussov, K. Smedt, T. Salmi), Sweden\\
    2014--\phantom{2020} & Co-advisor of 3 BSc and MSc students (J. Kuuttila, T. Salmi), Finland\\[1ex]
\end{tabular}
In addition, supervised various shorter student projects:
%In addition, Nordita host for visiting PhD. students:
%E. Van Woerkom,  2023;
E. van Woerkom (undergrad, 2023),
R. Serrano  (undergrad, 2023),
M. Bussov  (PhD, 2019),
K. Smedt  (PhD, 2019),
T. Salmi (PhD, 2019).

%\vspace{-0.3cm}
\section*{TEACHING}
\vspace{-0.3cm}
%Lectures (as a consultant) at the Finnish IT Center for Science (CSC) and universities/institutes. 
%Excellent student ratings on courses (e.g., 9.8/10 for CSC Summer Schools). \\[1.0ex]
\begin{tabular}{L!{\VRule}R}
2021        & Lecturer, Nordita Winter School: \textit{Waves in Astrophysics}, Nordita, Sweden \\
%            & \small{lecturer on ''Numerical methods for collisionless plasmas''.} \\[0ex]
2019        & Visiting Lecturer, \textit{Computational fluid dynamics}, Columbia University, USA \\
%            & \small{Visiting lecturer on ''Collisionless plasma simulations''.} \\[0ex]
2015--2019 & Lecturer, ($5\times$) \textit{High Performance Computing Summer School}, CSC, Finland \\
%    \footnotesize{(5 times)} & \small{Lecturer \& tutor for Finnish IT Center for Science HPC Summer School.} \\[0ex]
2018--2019  & Lecturer, ($2 \times$) \textit{Introduction to Julia}, CSC, Finland (course developed from scratch)\\
%    \footnotesize{(2 times)} & \small{Lecturer for an introductory course on the Julia programming language.} \\[0ex]
2015--2017 & Lecturer, ($3\times$) \textit{Software tools in Physics}, University of Turku, Finland \\
%    \footnotesize{(3 times)}  & \small{Lecturer of the ``Introduction to Unix'' section of the course (3 ECTS).} \\[1ex]
2011--2016 & Teaching Assistant in 10 courses (e.g., \textit{Thermophysics}, \textit{Electricity and Magnetism})
\end{tabular}
%In addition, teaching assistant in \textbf{Optics} (2016; 6 ECTS) in Univ. Turku, and \textbf{Thermophysics} (3 times, 2011---2013; 6 ECTS), \textbf{Electricity and Magnetism} (2012; 4 ECTS), \textbf{Laboratory Exercises in Physics 1} (2 times, 2011---2012; 3 ECTS), \textbf{Mathematics of Physics} (2011; 6 ECTS), and \textbf{Waveforms and Optics} (2 times, 2011; 6 ECTS) in Univ. Oulu.




\section*{TALKS, SEMINARS, AND COLLOQUIUMS}
\vspace{-0.3cm}
\noindent
In total, 2 colloquiums, 28 invited talks/seminars, and 31 contributed talks. Most recent ones include:\\[1.0ex]
\begin{tabular}{L!{\VRule}R}
%  2023 & Talk in \textit{Cosmic Connections} workshop, Flatiron Institute, USA \\
%  2023 & Talk in PCTS workshop \textit{Perspectives on Modern Numerical Methods}, Princeton, USA \\
%  2023 & IAS Coffee talk, IAS, USA \\
2023 & Invited talk and participation in Aspen Workshop on \textit{``Astro-bio-geo fluids''}, USA \\
  2023 & N3AS seminar, Berkeley (remote), USA \\
  2023 & CTC seminar at University of Maryland, USA \\
  2023 & Invited talk at Hamilton Institute Workshop on \textit{Relativistic Plasmas}, Ireland \\
%  2023 & Invited talk at CCA/Princeton Plasma Meeting,  USA \\
%  2022 & invited talk at CITA EOS group meeting, Canada \\
  2022 & Seminar at CITA (Canadian Institute for Theoretical Astrophysics), Canada \\
%  2022 & Talk at APS DPP, USA\\
  2022 & Colloquium at the Department of Physics, Brandeis University, USA \\
  2022 & Invited talk at ECT Workshop on ``\textit{Neutron stars as multimessenger laboratories}'', Italy\\
%  2022 & Invited talk at Nordita Program on ``\textit{Magnetic field evolution}'', Sweden\\
%  2022 & Talk at \textit{Purdue Plasma 2022} meeting, USA\\
  2022 & Nordita Astrophysics Seminar, Nordita, Sweden\\
%  2022 & Talk at \textit{Stars \& Compact Object} meeting, Flatiron Institute, USA \\
%  2022 & Talk at \textit{CCA Lunch talk}, Flatiron Institute, USA \\
%  2021 & Talk at \textit{Frontiers in Relativistic Astrophysics}, Flatiron Institute, USA \\
%  2021 & Talk at MIAPP Program on ``\textit{High Energy Plasma Phenomena}'', Germany \\
  2021 & Invited talk and participation in Aspen Workshop on \textit{"Exploring Extreme Matter"}, USA \\
%  2021 & \textbf{\textit{Invited:}} Frankfurt AstroCoffee Seminar, Frankfurt, Germany. \\
%  2020 & \textbf{\textit{Invited:}} IAS Coffee talk, Princeton, USA. \\
  2020 & Colloquium at Department of Physics, University of Helsinki, Finland \\
%  2020 & Invited seminar  on \textit{Nordic Dynamo Series}, Nordita, Sweden \\
%  2020 & Nordita Astro group meeting, Stockholm, Sweden. \\
%  2020 & Invited talk at \textit{St. Louis neutron star meeting}, U. of St. Louis, USA \\
%  2020 & CCA compact object meeting, New York, USA. \\
%  2019 & Extreme Objects Meeting, Stockholm, Sweden. \\
%  2019 & Invited seminar at Tartu Observatory, Estonia\\
%  2019 & KITP conference on Multiscale modeling of Plasmas, Santa Barbara, USA. \\
%  2019 & 2x Invited talks at Lorentz Center workshop on ``\textit{Bursting the Bubble}'', The Netherlands \\
%  2019 & Invited seminar at Department of Astronomy, University of T\"ubingen, Germany \\
%  2019 & \textbf{\textit{Invited:} Astro talk}, Uppsala University, Uppsala, Sweden. \\
%  2019 & \textbf{Extreme Objects Meeting}, Stockholm, Sweden. \\
%  2018 & \textbf{Astroplasmas: Particle acceleration and transport}, Rende, Italy. \\
%  2018 & \textbf{Tuorla-Tarto meeting}, Turku, Finland. \\
%  2018 & \textbf{\textit{Invited:}} \textbf{Time for Accretion}, Sigtuna, Sweden. \\
%  2018 & \textbf{Astroplasmas seminar}, Princeton, USA. \\
%  2018 & \textbf{High energy astro group meeting}, Columbia University, USA. \\
%  2018 & \textbf{Nordita Seminar}, Nordita, Sweden. \\
%  2018 & \textbf{\textit{Invited:}} \textbf{Astronomers' days}, Kuusamo, Finland. \\
%  2018 & \textbf{\textit{Invited:}} \textbf{Fire and Ice: Hot QCD meets cold and dense matter}, Saariselk\"a, Finland. \\
%  2017 & \textbf{\textit{Invited:}} \textbf{Holographic dense QCD and neutron stars}, ENS, Paris, France. \\
% 2017 & \textbf{Astrophysics Seminar}, Helsinki, Finland. \\
% 2017 & \textbf{Exascale thinking of particle energization problems}, Nordita, Sweden. \\
% 2016 & \textbf{From quarks to gravitational waves: Neutron stars as a laboratory for fundamental physic}, CERN. \\
% 2016 & \textbf{COSPAR 2016, E1.1: Accreting Neutron Stars and Stellar-mass Black Hole}, Istanbul, Turkey. (\textit{Conference canceled!})\\
% 2016 & \textbf{INT-16-2b: Phases Of Dense Matter Workshop}, Seattle, USA. \\
% 2016 & \textbf{JINA-CEE Symposium: Neutron Stars in the Multi-Messenger Era}, Ohio, USA. \\
% 2016 & \textbf{Nordita Workshop on accretion onto magnetized neutron stars}, Stockholm, Sweden. \\
% 2015 & \textbf{Workshop on Relativistic Astrophysics}, Kavalto, Finland. \\
% 2015 & \textbf{University of Maryland, Colloquium speaker}, Washington, USA.\\
% 2015 & \textbf{University of Tennessee, Colloquium speaker}, Tennessee, USA.\\
% 2015 & \textbf{The Neutron Star Radius, And All That Jazz}, Montreal, Canada.\\
% 2015 & \textbf{40 years of X-ray bursts: Extreme explosions in dense environments}, Madrid, Spain.\\
% 2014 & \textbf{ESAC (visiting scientist presentation)}, Madrid, Spain. \\
% 2014 & \textbf{Physics of Neutron Stars Conference}, St. Petersburg, Russia. \\
% 2014 & \textbf{Astronomers' Days}, Savonlinna, Finland. \\
% 2013 & \textbf{European Week of Astronomy and Space Science}, Turku, Finland. \\
% 2012 & \textbf{Astronomers' Days}, Porvoo, Finland. \\
\end{tabular}

\vspace{-0.3cm}
\section*{FUNDING}
\vspace{-0.3cm}

\begin{tabular}{L!{\VRule}R}
  2023  & $80\,000$ USD, NASA Fermi grant, Co-PI (PI: A. Beloborodov), Columbia University, USA\\
        & ``\textit{Gamma-ray precursors from neutron star mergers }''\\
  2021  & $300\,000$ USD, NASA ATP grant, Co-I (PI: A. Beloborodov), Columbia University, USA\\
        & ``\textit{Magnetic dissipation and radiation from compact objects}''\\
  2020  & $60\,000$ EUR, Wenner-Gren grant, Co-I (PI: D. Mitra), Nordita, Sweden\\
          & ``\textit{Reconnection, Radio observations, and switchbacks}''\\
  2020  & $300\,000$ USD, NASA ATP grant, Co-I (PI: L. Sironi), Columbia University, USA\\
        & ``\textit{Thermal and non-thermal emission in galaxy clusters: a first-principles approach}''\\
  2016  & $2\,000$ EUR, Magnus Ehrnrooth Foundation, travel grant, Finland \\
  2015 & $82\,000$ EUR, UTUGS Physical and Chemical Sciences PhD scholarship, Finland\\
%    & "Constraining neutron star mass and radius"\\[0.0ex]
  2015 & $23\,000$ EUR, Magnus Ehrnrooth Foundation PhD scholarship (declined), Finland\\
%    & "Magnetar atmosphere models" (\textit{Declined}) \\[0.0ex]
  2014 & $23\,000$ EUR, V\"ais\"al\"a Foundation PhD scholarship, Finland\\
%    & "Magnetar atmosphere models: breaking the barrier between observations and theory" \\[1.0ex]
\end{tabular}

\section*{SUPERCOMPUTING TIME AWARDS}
\vspace{-0.3cm}
\begin{tabular}{L!{\VRule}R}
    2022       & 1 MCPUhrs, Co-PI, CSC, \textit{Bayesian parameter constraints for neutron stars}, Finland\\
    2019--2023 & $\sim 20$ MCPUhrs/year, PI, internal Flatiron Institute supercomputers, USA\\
    2018--2021 & $\sim 20$ MCPUhrs/year, Co-PI, SNIC, \textit{Astrophysical turbulence and dynamo action}, Sweden\\
    2018       & $\sim 60$ kCPUhrs, PI, SNIC/HPC2N, \textit{Relativistic plasmas in silico}, Sweden\\
\end{tabular}

%\section*{PUBLICATION RECORD}
%\vspace{-0.3cm}
%\noindent
%I have 30 original research articles published in international peer-reviewed journals, including 
%\textit{Nature Physics}, 
%\textit{Physical Review Letters}, and 
%\textit{Physical Review X}. 
%My total citation count is 1076 in \href{https://ui.adsabs.harvard.edu/search/q=%20author%3A%22nattila%22&sort=date%20desc%2C%20bibcode%20desc&p_=0}{ADS}
%(1205 in \href{https://scholar.google.com/citations?user=d1fD9oYAAAAJ&hl=en}{Google Scholar}); 
%h-index is 18, i10-index 21, and i100-index 3. 



\section*{INSTITUTIONAL RESPONSIBILITIES}
\vspace{-0.3cm}
\begin{tabular}{L!{\VRule}R}
    2021--\phantom{2000} & Main organizer, \textit{Flatiron Observational Astrophysics Series}, USA\\
    2021--\phantom{2000} & Member, Open Science Working Group (sub-group under Young Academy Finland)\\
    2017\phantom{--2000} & Student Member, Astronomy Faculty Search Committee, University of Turku, Finland\\
    2017\phantom{--2000} & Convenor, \textit{Computational Coffee Break}, Tuorla Observatory, Finland\\
    2016--\phantom{2000} & Member, eXTP Science Working Group, Dense Matter\\
    2015--2018           & Member, ESA XIPE Satellite Science Team (SWG2.2 Accreting Millisecond Pulsars)\\
    2014--2019           & Member, organizing committee, CSC HPC Summer Schools, CSC, Finland\\
    2013--2019           & Member, JuliaLang (open-source organization for Julia programming language)\\
\end{tabular}


\section*{CONFERENCE ORGANIZATION}
\vspace{-0.3cm}
\begin{tabular}{L!{\VRule}R}
    2023 & Lead organizer, workshop, \textit{Black Hole Flares: Connecting Theory and Observations}, USA\\
    2023 & Lead organizer, workshop, \textit{Astrophysics of Fast Radio Bursts II}, USA\\
    2022 & Organizer, workshop, \textit{Dynamics of Coherent Structures in Astro-Geo-Turbulence}, USA\\
    2022 & Co-organizer, symposium, \textit{Flatiron Exoplanet Symposium}, USA\\
    2022 & Co-organizer, workshop, \textit{Physics of Exoplanet Atmospheres}, USA\\
    2022 & Co-organizer, PCTS workshop, \textit{Weather and Climate on Neutron Stars}, USA\\
    2021 & Lead organizer, workshop, \textit{Frontiers in Relativistic Turbulence}, USA\\
    2017 & Co-organizer, workshop, \textit{Exascale thinking to Particle Energization Problems}, Sweden\\
    2015 & Local organizing committee, workshop, \textit{Relativistic Astrophysics}, Finland\\
\end{tabular}



%\vspace{-0.3cm}
\section*{REVIEWING ACTIVITIES}
\vspace{-0.3cm}

%\textbf{Peer reviewer:} Nature, Physical Review Letters, The Astrophysical Journal Letters, The Astrophysical Journal, The Astrophysical Journal Supplement Series, Monthly Notices of the Royal Astronomical Society, Astronomy \& Astrophysics, Physical Review D, Physical Review E, European Physical Journal A, Universe.

\noindent
\textbf{Peer reviewer:} Nature, PRL, ApJL, ApJ, ApJS, MNRAS, A\&A, PRD, PRE, EPJA, Universe

\noindent
\textbf{Grant reviewer:} DOE (Fusion energy sciences)

%\vspace{-0.3cm}
\section*{MEMBERSHIPS OF SCIENTIFIC SOCIETIES}
\vspace{-0.3cm}
\begin{tabular}{L!{\VRule}R}
    2019--\phantom{3000}& Member, Young Academy Finland, under Academy of Science and Letters, Finland\\
    2018--\phantom{3000}& IAU Junior Member\\
    2012--\phantom{3000}& Member, Finnish Astronomical Society, Finland\\
\end{tabular}

\vspace{-0.3cm}
\section*{PUBLIC OUTREACH}
\vspace{-0.3cm}
\begin{tabular}{L!{\VRule}R}
    2020--\phantom{3000}& Responses/commentaries on newspapers and popular science magazines\\
                        & (T\"ahdet \& Avaruus, 9/2020, 10/2020; Helsingin Sanomat 6/1/2021)\\
    2020\phantom{--3000}& Meet the Scientist, \href{https://www.youtube.com/watch?v=Ch38VpF341I}{youtu.be/Ch38VpF341I}\\
                        & Educational video about \textit{Gravitation} for high-school students\\
    2019\phantom{--3000}& Public Science Talk, Academy Club for Young Scientists, \href{https://www.youtube.com/watch?v=W7ljVlSEAX4}{youtu.be.com/W7ljVlSEAX4} \\
                        & \textit{Astrophysical Turbulence: from stirring coffee to mixing galaxies} \\
    2019--2020          & Appearances on popular science articles on \textit{Quark matter cores in neutron stars}\\
                        & \href{https://astrobites.org/2019/03/29/a-strange-type-of-matter-may-lie-at-the-heart-of-neutron-stars/}{Astrobites},
         \href{https://www.universetoday.com/146476/neutron-stars-could-have-a-layer-of-exotic-quark-matter-inside-them/}{Universe Today},
         \href{https://physicsworld.com/a/neutron-stars-may-contain-free-quarks/}{Physics World},
         \href{https://en.wikipedia.org/wiki/QCD_matter}{Wikipedia},
         T\"ahdet \& Avaruus 4/2019 \\
    2018\phantom{--3000} & Personal profile on Finnish astronomy Magazine T\"ahdet \& Avaruus 2/2018\\
    2017\phantom{--3000} & Appearances on popular newspaper articles on \textit{Groundbreaking neutron star measurement}\\
                         & Incl. 
                         \href{https://cosmosmagazine.com/space/nuke-blasts-reveal-true-size-of-neutron-stars}{Cosmos},
                         \href{https://phys.org/news/2017-11-method-neutron-star-size-based.html}{Phys.org}.
  \href{https://www.avaruus.fi/uutiset/tahdet-sumut-ja-galaksit/turkulaiset-keksivat-uuden-tavan-mitata-neutronitahtien-kokoa.html}{T{\"a}hdet \& Avaruus (25.11.2017)}\\
%  \href{https://www.turkulainen.fi/artikkeli/578926-turun-yliopiston-tutkimusryhma-kehitti-tavan-mitata-neutronitahtien-kokoa}{Turkulainen (10.11.2017)},
%  \href{http://www.ts.fi/uutiset/paikalliset/3724265/uusi+menetelma+mahdollistaa+neutronitahtien+sateen+mittauksen+galaksin+toiselta+laidalta}{Turun Sanomat (10.11.2017)},
%  \href{http://www.aamuset.fi/uutiset/3758822/kosmiset+ydinrajahdykset+tuovat+uutta+tietoa+neutronitahtien+rakenteesta}{Aamuset (8.12.2017)},
%  \href{https://www.tekniikkatalous.fi/tiede/avaruus/neutronitahtien-tutkija-kaytti-apunaan-nasa-n-satelliitteja-kynaa-ja-paperia-kuutiosentti-neutronimateriaa-painaa-uskomattomat-100-miljoonaa-tonnia-6691137}{Tekniikka \& Talous (8.12.2017)},
%  \href{https://www.verkkouutiset.fi/kosmisista-ydinrajahdyksista-uutta-tietoa-neutronitahtien-rakenteesta/}{Verkkouutiset (8.12.2017)} \\
\end{tabular}


\vspace{-0.3cm}
\section*{MAJOR COLLABORATIONS}
\vspace{-0.3cm}
\textbf{A. Beloborodov} (radiative plasmas, Columbia, USA), 
\textbf{A. Brandenburg} (MHD turbulence, Nordita, Sweden), 
\textbf{J.Y-K Cho} (exoplanet fluid dynamics, Brandeis, USA), 
%\textbf{L. Comisso} (relativistic turbulence, Columbia, USA), 
%\textbf{J. Kajava} (thermonuclear X-ray bursts, ESAC, Spain), 
\textbf{A. Kurkela}, (neutron star quark matter, Stavanger, Norway),  
\textbf{C. Miller} (neutron star mass-radius measurements, Maryland, USA), 
\textbf{D. Mitra} (switchbacks in the solar wind, Nordita, Sweden), 
\textbf{A. Philippov} (pulsar magnetospheres, Maryland, USA),  
%\textbf{J. Poutanen} (pulse profile modeling, Turku, Finland), 
%\textbf{B. Ripperda} (compact object magnetospheres, IAS, USA), 
\textbf{L. Sironi} (turbulence, reconnection, and shocks, Columbia, USA),  
\textbf{A. Steiner} (neutron star EOS, Tennessee USA), 
\textbf{V. Suleimanov} (neutron star atmosphere models, Tubingen, Germany), 
%\textbf{A. Veledina} (accretion flows, Turku, Finland), 
\textbf{A. Vuorinen} (neutron star quark matter, Helsinki, Finland), 
%\textbf{V. Zhdankin} (turbulence, Flatiron, USA).


%--------------------------------------------------
\clearpage

\setcounter{page}{1}
\renewcommand\refname{\phantom{bla}}


%Ruma kun mikä, mutta toimii
%Jani Lappalainen 2011

\newcommand{\aj}{AJ}%
          % Astronomical Journal
\newcommand{\actaa}{Acta Astron.}%
          % Acta Astronomica
\newcommand{\araa}{ARA\&A}%
          % Annual Review of Astron and Astrophys
\newcommand{\apj}{ApJ}%
          % Astrophysical Journal
\newcommand{\apjl}{ApJ}%
          % Astrophysical Journal, Letters
\newcommand{\apjs}{ApJS}%
          % Astrophysical Journal, Supplement
\newcommand{\ao}{Appl.~Opt.}%
          % Applied Optics
\newcommand{\apss}{Ap\&SS}%
          % Astrophysics and Space Science
\newcommand{\aap}{A\&A}%
          % Astronomy and Astrophysics
\newcommand{\aapr}{A\&A~Rev.}%
          % Astronomy and Astrophysics Reviews
\newcommand{\aaps}{A\&AS}%
          % Astronomy and Astrophysics, Supplement
\newcommand{\azh}{AZh}%
          % Astronomicheskii Zhurnal
\newcommand{\baas}{BAAS}%
          % Bulletin of the AAS
\newcommand{\bac}{Bull. astr. Inst. Czechosl.}%
          % Bulletin of the Astronomical Institutes of Czechoslovakia 
\newcommand{\caa}{Chinese Astron. Astrophys.}%
          % Chinese Astronomy and Astrophysics
\newcommand{\cjaa}{Chinese J. Astron. Astrophys.}%
          % Chinese Journal of Astronomy and Astrophysics
\newcommand{\icarus}{Icarus}%
          % Icarus
\newcommand{\jcap}{J. Cosmology Astropart. Phys.}%
          % Journal of Cosmology and Astroparticle Physics
\newcommand{\jrasc}{JRASC}%
          % Journal of the RAS of Canada
\newcommand{\mnras}{MNRAS}%
          % Monthly Notices of the RAS
\newcommand{\memras}{MmRAS}%
          % Memoirs of the RAS
\newcommand{\na}{New A}%
          % New Astronomy
\newcommand{\nar}{New A Rev.}%
          % New Astronomy Review
\newcommand{\pasa}{PASA}%
          % Publications of the Astron. Soc. of Australia
\newcommand{\pra}{Phys.~Rev.~A}%
          % Physical Review A: General Physics
\newcommand{\prb}{Phys.~Rev.~B}%
          % Physical Review B: Solid State
\newcommand{\prc}{Phys.~Rev.~C}%
          % Physical Review C
\newcommand{\prd}{Phys.~Rev.~D}%
          % Physical Review D
\newcommand{\pre}{Phys.~Rev.~E}%
          % Physical Review E
\newcommand{\prl}{Phys.~Rev.~Lett.}%
          % Physical Review Letters
\newcommand{\pasp}{PASP}%
          % Publications of the ASP
\newcommand{\pasj}{PASJ}%
          % Publications of the ASJ
\newcommand{\qjras}{QJRAS}%
          % Quarterly Journal of the RAS
\newcommand{\rmxaa}{Rev. Mexicana Astron. Astrofis.}%
          % Revista Mexicana de Astronomia y Astrofisica
\newcommand{\skytel}{S\&T}%
          % Sky and Telescope
\newcommand{\solphys}{Sol.~Phys.}%
          % Solar Physics
\newcommand{\sovast}{Soviet~Ast.}%
          % Soviet Astronomy
\newcommand{\ssr}{Space~Sci.~Rev.}%
          % Space Science Reviews
\newcommand{\zap}{ZAp}%
          % Zeitschrift fuer Astrophysik
\newcommand{\nat}{Nature}%
          % Nature
\newcommand{\iaucirc}{IAU~Circ.}%
          % IAU Cirulars
\newcommand{\aplett}{Astrophys.~Lett.}%
          % Astrophysics Letters
\newcommand{\apspr}{Astrophys.~Space~Phys.~Res.}%
          % Astrophysics Space Physics Research
\newcommand{\bain}{Bull.~Astron.~Inst.~Netherlands}%
          % Bulletin Astronomical Institute of the Netherlands
\newcommand{\fcp}{Fund.~Cosmic~Phys.}%
          % Fundamental Cosmic Physics
\newcommand{\gca}{Geochim.~Cosmochim.~Acta}%
          % Geochimica Cosmochimica Acta
\newcommand{\grl}{Geophys.~Res.~Lett.}%
          % Geophysics Research Letters
\newcommand{\jcp}{J.~Chem.~Phys.}%
          % Journal of Chemical Physics
\newcommand{\jgr}{J.~Geophys.~Res.}%
          % Journal of Geophysics Research
\newcommand{\jqsrt}{J.~Quant.~Spec.~Radiat.~Transf.}%
          % Journal of Quantitiative Spectroscopy and Radiative Trasfer
\newcommand{\memsai}{Mem.~Soc.~Astron.~Italiana}%
          % Mem. Societa Astronomica Italiana
\newcommand{\nphysa}{Nucl.~Phys.~A}%
          % Nuclear Physics A
\newcommand{\physrep}{Phys.~Rep.}%
          % Physics Reports
\newcommand{\physscr}{Phys.~Scr}%
          % Physica Scripta
\newcommand{\planss}{Planet.~Space~Sci.}%
          % Planetary Space Science
\newcommand{\procspie}{Proc.~SPIE}%
          % Proceedings of the SPIE

\begin{center}
\textbf{PUBLICATION RECORD}
\end{center}

\noindent
I have 33 original research articles published in international peer-reviewed journals, including 
\textit{Nature Physics}, 
\textit{Physical Review Letters}, and 
\textit{Physical Review X}. 
In addition, 1 invited book chapter, 3 conference proceedings, and 6 major open-source software repositories.\\

\noindent
My total citation count is 1374 in \href{https://ui.adsabs.harvard.edu/search/q=%20author%3A%22nattila%22&sort=date%20desc%2C%20bibcode%20desc&p_=0}{ADS}
(1539 in \href{https://scholar.google.com/citations?user=d1fD9oYAAAAJ&hl=en}{Google Scholar}); 
h-index is 18, i10-index 24, and i100-index 3. 


\section*{Submitted:}
\vspace{-1cm}

\begin{thebibliography}{4}

\bibitem{2021arXiv211008024y}
\textbf{J}.~{\textbf{N\"attil\"a}}, J.~Y-K. {Cho}, J.W. Skinner, E.R. Most, B. Ripperda.
\newblock {Neutron Star Atmosphere-Ocean Dynamics}.
\newblock {\em ApJ}, July 2023.
\href{http://arxiv.org/abs/2306.08186}{\nolinkurl{ [arXiv:2306.08186]}}.

\bibitem{xxx}
E.~{Annala}, T.~{Gorda}, A.~{Kurkela}, \textbf{J.} \textbf{{N{\"a}ttil{\"a}}},
  and A.~{Vuorinen}.
\newblock {Strongly interacting matter exhibits deconfined behavior in massive neutron stars}.
\newblock {\em }, December 2022.
\href{http://arxiv.org/abs/2303.11356}{\nolinkurl{ [arXiv:2303.11356]}}.

\bibitem{2021arXiv211008024x}
J.~{Skinner}, \textbf{J}.~{\textbf{N\"attil\"a}}, and J.~Y-K. {Cho}.
\newblock {Repeated Cyclogenesis on Hot-Exoplanet Atmospheres with Deep Heating}.
\newblock {\em Phys. Rev. Lett.}, December 2022.
\href{http://arxiv.org/abs/2212.05114}{\nolinkurl{ [arXiv:2212.05114]}}.


\end{thebibliography}



\section*{Published:}
\vspace{-1cm}

\begin{thebibliography}{10}


\bibitem{ns_book}
\textbf{J. N\"attil\"a} and J.~J.~E. {Kajava}.
\newblock {Fundamental Physics with Neutron Stars}.
\newblock {\em Handbook of X-ray and Gamma-ray Astrophysics }
\newblock (editors: Cosimo Bambi and Andrea Santangelo), December 2022, Springer.
\href{http://arxiv.org/abs/2211.15721}{\nolinkurl{ [arXiv:2211.15721]}}.

\bibitem{2021arXiv211008024z}
C.~{Demidem}, \textbf{J}.~{\textbf{N\"attil\"a}}, and A.~{Veledina}.
\newblock {Relativistic Collisionless Shocks in Inhomogeneous Magnetized Plasmas}.
\newblock {\em ApJL}, December 2022.
\href{http://arxiv.org/abs/2212.06053}{\nolinkurl{ [arXiv:2212.06053]}}.

\bibitem{2021arXiv211008024S}
K.~{Smedt}, D.~{Ruprecht}, J.~{Niesen}, S.~{Tobias}, and \textbf{J}.
  {\textbf{N{\"a}ttil{\"a}}}.
\newblock {New applications for the Boris Spectral Deferred Correction
  algorithm for plasma simulations}.
\newblock {\em Applied Mathematics and Computation}, November 2022,
  \href{http://arxiv.org/abs/2110.08024}{\nolinkurl{ [arXiv:2110.08024]}}.

\bibitem{2022PhRvL.128g5101N}
\textbf{J}. \textbf{N{\"a}ttil{\"a}} and A.~M. {Beloborodov}.
\newblock {Heating of Magnetically Dominated Plasma by Alfv{\'e}n-Wave
  Turbulence}.
\newblock {\em \prl}, 128(7):075101, February 2022,
  \href{http://arxiv.org/abs/2111.15578}{\nolinkurl{ [arXiv:2111.15578]}}.


\bibitem{2021arXiv210901381B}
M.~{Bussov} and \textbf{J.} {\textbf{N{\"a}ttil{\"a}}}.
\newblock {Segmentation of turbulent computational fluid dynamics simulations
  with unsupervised ensemble learning}.
\newblock {\em Signal Processing: Image Communication}, 99:116450, September
  2021, \href{http://arxiv.org/abs/2109.01381}{\nolinkurl{
  [arXiv:2109.01381]}}.

\bibitem{2021PhRvL.127c5101S}
L.~{Sironi}, I.~{Plotnikov}, \textbf{J. {N{\"a}ttil{\"a}}}, and A.~M.
  {Beloborodov}.
\newblock {Coherent Electromagnetic Emission from Relativistic Magnetized
  Shocks}.
\newblock {\em \prl}, 127(3):035101, July 2021,
  \href{http://arxiv.org/abs/2107.01211}{\nolinkurl{ [arXiv:2107.01211]}}.

\bibitem{2021arXiv210505132A}
E.~{Annala}, T.~{Gorda}, E.~{Katerini}, A.~{Kurkela}, \textbf{J.
  {N{\"a}ttil{\"a}}}, V.~{Paschalidis}, and A.~{Vuorinen}.
\newblock {Multimessenger constraints for ultra-dense matter}.
\newblock {\em \textit{PRX}}, May 2021,
  \href{http://arxiv.org/abs/2105.05132}{\nolinkurl{ [arXiv:2105.05132]}}.

\bibitem{2021MNRAS.503..688S}
E.~{Sobacchi}, \textbf{J. {N{\"a}ttil{\"a}}}, and L.~{Sironi}.
\newblock {A fully kinetic model for orphan gamma-ray flares in blazars}.
\newblock {\em \mnras}, 503(1):688--693, May 2021,
  \href{http://arxiv.org/abs/2102.11770}{\nolinkurl{ [arXiv:2102.11770]}}.

\bibitem{2021ApJ...921...87N}
\textbf{J}. \textbf{N{\"a}ttil{\"a}} and A.~M. {Beloborodov}.
\newblock {Radiative Turbulent Flares in Magnetically Dominated Plasmas}.
\newblock {\em \apj}, 921(1):87, November 2021,
  \href{http://arxiv.org/abs/2012.03043}{\nolinkurl{ [arXiv:2012.03043]}}.

\bibitem{2020arXiv200812817A}
M.~{Al-Mamun}, A.~W. {Steiner}, \textbf{J.} \textbf{N{\"a}ttil{\"a}},
  J.~{Lange}, R.~{O'Shaughnessy}, I.~{Tews}, S.~{Gandolfi}, C.~{Heinke}, and
  S.~{Han}.
\newblock {Combining Electromagnetic and Gravitational-Wave Constraints on
  Neutron-Star Masses and Radii}.
\newblock {\em \prl}, August 2020,
  \href{http://arxiv.org/abs/2008.12817}{\nolinkurl{ [arXiv:2008.12817]}}.

\bibitem{2020A&A...643A..84L}
V.~{Loktev}, T.~{Salmi}, \textbf{J}. \textbf{N{\"a}ttil{\"a}}, and
  J.~{Poutanen}.
\newblock {Oblate Schwarzschild approximation for polarized radiation from
  rapidly rotating neutron stars}.
\newblock {\em \aap}, 643:A84, November 2020,
  \href{http://arxiv.org/abs/2009.08852}{\nolinkurl{ [arXiv:2009.08852]}}.

\bibitem{2020A&A...641A..15S}
T.~{Salmi}, V.~F. {Suleimanov}, \textbf{J}. \textbf{N{\"a}ttil{\"a}}, and
  J.~{Poutanen}.
\newblock {Magnetospheric return-current-heated atmospheres of rotation-powered
  millisecond pulsars}.
\newblock {\em \aap}, 641:A15, September 2020,
  \href{http://arxiv.org/abs/2002.11427}{\nolinkurl{ [arXiv:2002.11427]}}.

\bibitem{2020NatPh..16..907A}
E.~{Annala}, T.~{Gorda}, A.~{Kurkela}, \textbf{J.} \textbf{{N{\"a}ttil{\"a}}},
  and A.~{Vuorinen}.
\newblock {Evidence for quark-matter cores in massive neutron stars}.
\newblock {\em Nature Physics}, 16(9):907--910, June 2020,
  \href{http://arxiv.org/abs/1903.09121}{\nolinkurl{ [arXiv:1903.09121]}}.

\bibitem{2020A&A...638A.142A}
P.~{Abolmasov}, \textbf{J}. \textbf{N{\"a}ttil{\"a}}, and J~{Poutanen}.
\newblock {Kilohertz quasi-periodic oscillations from neutron star spreading
  layers}.
\newblock {\em \aap}, 638:A142, June 2020,
  \href{http://arxiv.org/abs/1910.09906}{\nolinkurl{ [arXiv:1910.09906]}}.

\bibitem{2019ApJ...884..144V}
A.~{Veledina}, \textbf{J}. \textbf{N{\"a}ttil{\"a}}, and A.~M. {Beloborodov}.
\newblock {Pulsar Wind-heated Accretion Disk and the Origin of Modes in
  Transitional Millisecond Pulsar PSR J1023+0038}.
\newblock {\em \apj}, 884(2):144, October 2019,
  \href{http://arxiv.org/abs/1906.02519}{\nolinkurl{ [arXiv:1906.02519]}}.

\bibitem{2019A&A...629A..89N}
F.~{Nauman} and \textbf{J}. \textbf{N{\"a}ttil{\"a}}.
\newblock {Exploring helical dynamos with machine learning: Regularized linear
  regression outperforms ensemble methods}.
\newblock {\em \aap}, 629:A89, September 2019,
  \href{http://arxiv.org/abs/1905.08193}{\nolinkurl{ [1905.08193]}}.

\bibitem{2019arXiv190606306N}
\textbf{J}. \textbf{N{\"a}ttil{\"a}}.
\newblock {Runko: Modern multiphysics toolbox for plasma simulations}.
\newblock {\em A\&A}, 664:A68, August 2022 (submitted originally on June 2019),
  \href{http://arxiv.org/abs/1906.06306}{\nolinkurl{ [1906.06306]}}.

\bibitem{2019SCPMA..6229506I}
J.~J.~M. {in't Zand}, E.~{Bozzo}, J.~{Qu}, X.-D. {Li}, L.~{Amati}, Y.~{Chen},
  I.~{Donnarumma}, V.~{Doroshenko}, S.~A. {Drake}, and {et~al. (incl.
  \textbf{J}. \textbf{{N{\"a}ttil{\"a}}})}.
\newblock {Observatory science with eXTP}.
\newblock {\em Science China Physics, Mechanics, and Astronomy}, 62:29506,
  February 2019.

\bibitem{2019SCPMA..6229503W}
A.~L. {Watts}, W.~{Yu}, J.~{Poutanen}, S.~{Zhang}, S.~{Bhattacharyya},
  S.~{Bogdanov}, L.~{Ji}, A.~{Patruno}, T.~E. {Riley}, and {et~al. (incl.
  \textbf{J}. \textbf{{N{\"a}ttil{\"a}}})}.
\newblock {Dense matter with eXTP}.
\newblock {\em Science China Physics, Mechanics, and Astronomy}, 62:29503,
  February 2019.

\bibitem{2018ApJ...866...53L}
Z.~{Li}, V.~F. {Suleimanov}, J.~{Poutanen}, T.~{Salmi}, M.~{Falanga},
  \textbf{J}. {\textbf{N{\"a}ttil{\"a}}}, and R.~{Xu}.
\newblock {Evidence for the Photoionization Absorption Edge in a Photospheric
  Radius Expansion X-Ray Burst from GRS 1747$-$312 in Terzan 6}.
\newblock {\em \apj}, 866:53, October 2018,
  \href{http://arxiv.org/abs/1809.00098}{\nolinkurl{ [arXiv:1809.00098]}}.

\bibitem{2018A&A...618A.161S}
T.~{Salmi}, \textbf{J}. \textbf{{N{\"a}ttil{\"a}}}, and J.~{Poutanen}.
\newblock {Bayesian parameter constraints for neutron star masses and radii
  using X-ray timing observations of accretion-powered millisecond pulsars}.
\newblock {\em \aap}, 618:A161, October 2018,
  \href{http://arxiv.org/abs/1805.01149}{\nolinkurl{ [arXiv:1805.01149]}}.

\bibitem{2018ApJ...863....8P}
P.~{Pihajoki}, M.~{Mannerkoski}, \textbf{J}. \textbf{{N{\"a}ttil{\"a}}}, and
  P.~H. {Johansson}.
\newblock {General purpose ray-tracing and polarized radiative transfer in
  General Relativity}.
\newblock {\em \apj}, 863:8, August 2018,
  \href{http://arxiv.org/abs/1804.04670}{\nolinkurl{ [arXiv:1804.04670]}}.

\bibitem{2018A&A...615A..50N}
\textbf{J}. \textbf{{N{\"a}ttil{\"a}}} and P.~{Pihajoki}.
\newblock {Radiation from rapidly rotating oblate neutron stars}.
\newblock {\em \aap}, 615:A50, July 2018,
  \href{http://arxiv.org/abs/1709.07292}{\nolinkurl{ [arXiv:1709.07292]}}.

\bibitem{2017A&A...608A..31N}
\textbf{J}. \textbf{{N{\"a}ttil{\"a}}}, M.~C. {Miller}, A.~W. {Steiner},
  J.~J.~E. {Kajava}, V.~F. {Suleimanov}, and J.~{Poutanen}.
\newblock {Neutron star mass and radius measurements from atmospheric model
  fits to X-ray burst cooling tail spectra}.
\newblock {\em \aap}, 608:A31, December 2017,
  \href{http://arxiv.org/abs/1709.09120}{\nolinkurl{ [arXiv:1709.09120]}}.

\bibitem{2017MNRAS.472.3905S}
V.~F. {Suleimanov}, J.~J.~E. {Kajava}, S.~V. {Molkov}, \textbf{J}.
  \textbf{{N{\"a}ttil{\"a}}}, A.~A. {Lutovinov}, K.~{Werner}, and
  J.~{Poutanen}.
\newblock {Basic parameters of the helium-accreting X-ray bursting neutron star
  in 4U 1820-30}.
\newblock {\em \mnras}, 472:3905--3913, December 2017,
  \href{http://arxiv.org/abs/1708.09168}{\nolinkurl{ [arXiv:1708.09168]}}.

\bibitem{2017MNRAS.472...78K}
J.~J.~E. {Kajava}, K.~I.~I. {Koljonen}, \textbf{J}. \textbf{N{\"a}ttil{\"a}},
  V.~{Suleimanov}, and J.~{Poutanen}.
\newblock {Variable spreading layer in 4U 1608-52 during thermonuclear X-ray
  bursts in the soft state}.
\newblock {\em \mnras}, 472:78--89, November 2017,
  \href{http://arxiv.org/abs/1707.09479}{\nolinkurl{ [arXiv:1707.09479]}}.

\bibitem{2017A&A...604A..77K}
J.~{Kuuttila}, J.~J.~E. {Kajava}, \textbf{J}. \textbf{{N{\"a}ttil{\"a}}}, S.~E.
  {Motta}, C.~{S{\'a}nchez-Fern{\'a}ndez}, E.~{Kuulkers}, A.~{Cumming}, and
  J.~{Poutanen}.
\newblock {Flux decay during thermonuclear X-ray bursts analysed with the
  dynamic power-law index method}.
\newblock {\em \aap}, 604:A77, August 2017,
  \href{http://arxiv.org/abs/1705.05653}{\nolinkurl{ [arXiv:1705.05653]}}.

\bibitem{2017MNRAS.466..906S}
V.~F. {Suleimanov}, J.~{Poutanen}, \textbf{J}. \textbf{N{\"a}ttil{\"a}},
  J.~J.~E. {Kajava}, M.~G. {Revnivtsev}, and K.~{Werner}.
\newblock {The direct cooling tail method for X-ray burst analysis to constrain
  neutron star masses and radii}.
\newblock {\em \mnras}, 466:906--913, April 2017,
  \href{http://arxiv.org/abs/1611.09885}{\nolinkurl{ [arXiv:1611.09885]}}.

\bibitem{2017MNRAS.464L...6K}
J.~J.~E. {Kajava}, \textbf{J}. \textbf{N{\"a}ttil{\"a}}, J.~{Poutanen},
  A.~{Cumming}, V.~{Suleimanov}, and E.~{Kuulkers}.
\newblock {Detection of burning ashes from thermonuclear X-ray bursts}.
\newblock {\em \mnras}, 464:L6--L10, January 2017,
  \href{http://arxiv.org/abs/1608.06801}{\nolinkurl{ [arXiv:1608.06801]}}.

\bibitem{2016A&A...591A..25N}
\textbf{J}. \textbf{N{\"a}ttil{\"a}}, A.~W. {Steiner}, J.~J.~E. {Kajava}, V.~F.
  {Suleimanov}, and J.~{Poutanen}.
\newblock {Equation of state constraints for the cold dense matter inside
  neutron stars using the cooling tail method}.
\newblock {\em \aap}, 591:A25, June 2016,
  \href{http://arxiv.org/abs/1509.06561}{\nolinkurl{ [arXiv:1509.06561]}}.

\bibitem{2015A&A...581A..83N}
\textbf{J}. \textbf{N{\"a}ttil{\"a}}, V.~F. {Suleimanov}, J.~J.~E. {Kajava},
  and J.~{Poutanen}.
\newblock {Models of neutron star atmospheres enriched with nuclear burning
  ashes}.
\newblock {\em \aap}, 581:A83, September 2015,
  \href{http://arxiv.org/abs/1507.01525}{\nolinkurl{ [arXiv:1507.01525]}}.

\bibitem{2014MNRAS.445.4218K}
J.~J.~E. {Kajava}, \textbf{J}. \textbf{N{\"a}ttil{\"a}}, O.-M. {Latvala},
  M.~{Pursiainen}, J.~{Poutanen}, V.~F. {Suleimanov}, M.~G. {Revnivtsev},
  E.~{Kuulkers}, and D.~K. {Galloway}.
\newblock {The influence of accretion geometry on the spectral evolution during
  thermonuclear (type I) X-ray bursts}.
\newblock {\em \mnras}, 445:4218--4234, December 2014,
  \href{http://arxiv.org/abs/1406.0322}{\nolinkurl{ [arXiv:1406.0322]}}.

\bibitem{2014MNRAS.442.3777P}
J.~{Poutanen}, \textbf{J}. \textbf{N{\"a}ttil{\"a}}, J.~J.~E. {Kajava}, O.-M.
  {Latvala}, D.~K. {Galloway}, E.~{Kuulkers}, and V.~F. {Suleimanov}.
\newblock {The effect of accretion on the measurement of neutron star mass and
  radius in the low-mass X-ray binary 4U 1608-52}.
\newblock {\em \mnras}, 442:3777--3790, August 2014,
  \href{http://arxiv.org/abs/1405.2663}{\nolinkurl{ [arXiv:1405.2663]}}.

\end{thebibliography}





\vspace{1cm}
\section*{Proceedings}

\begin{thebibliography}{1}
\vspace{-1cm}

\bibitem{SNOWMASS}
S. Bogdanov, et. al (including \textbf{J. N\"attil\"a}).
\newblock {Snowmass 2021 Cosmic Frontier White Paper: The Dense Matter Equation of State and QCD Phase Transitions}.
\newblock September, 2022.
[\href{https://arxiv.org/abs/2209.07412}{\nolinkurl{arXiv:2209.07412}}].

\bibitem{INTEGRAL}
E. {Annala}, T. {Gorda}, A. {Kurkela}, \textbf{J. {N{\"a}ttil{\"a}}}, and A. {Vuorinen}.
\newblock {Constraining the properties of neutron-star matter with observations}.
\newblock In {\em 12th INTEGRAL Conference}, Geneva, Switzerland, 11-15 February 2019
[\href{https://arxiv.org/abs/1904.01354}{\nolinkurl{arXiv:1904.01354}}].

\bibitem{XIPE}
P.~{Soffitta}, R.~{Bellazzini}, E.~{Bozzo}, V.~{Burwitz}, A.~{Castro-Tirado},
  E.~{Costa}, T.~{Courvoisier}, H.~{Feng}, S.~{Gburek}, R.~{Goosmann}, and
  et~al. (incl. \textbf{J}. \textbf{{N{\"a}ttil{\"a}}} )
\newblock {XIPE: the x-ray imaging polarimetry explorer}.
\newblock In {\em Space Telescopes and Instrumentation 2016: Ultraviolet to
  Gamma Ray}, volume 9905 Proceedings, page 990515, July 2016.
\href{https://doi.org/10.1117/12.2233046}{\nolinkurl{doi.org/10.1117/12.2233046}}.
% TODO
\end{thebibliography}



\vspace{1cm}
\section*{Theses}
\vspace{-1cm}

%\bibliographystyle{hunsrt}
\begin{thebibliography}{3}

\bibitem{NatjThesis}
\textbf{J. N\"attil\"a}.
\newblock{X-ray bursts as a tool to constrain the equation of state of the ultra-dense matter inside neutron stars}.
\newblock PhD thesis, University of Turku, Finland, 2017. \href{http://urn.fi/URN:ISBN:978-951-29-7057-5}{\nolinkurl{ISBN:978-951-29-7057-5}}.

\bibitem{NatjMaster}
\textbf{J. N\"attil\"a}.
\newblock {Mass and radius constraints for neutron stars using the cooling tail method}.
\newblock Master's thesis, University of Oulu, Finland, 2013. \href{http://urn.fi/URN:NBN:fi:oulu-201312041966}{\nolinkurl{oulu-201312041966}}.

\bibitem{NatjBachelor}
\textbf{J. N\"attil\"a}.
\newblock {Spectral analysis of X-ray bursts from neutron stars: IGR
  J1747--2721 (\textit{Neutronit\"ahtien r\"ontgenpurkaukset ja niiden
  spektrianalyysi: IGR J1747--2721})}.
\newblock Bachelor's thesis, University of Oulu, Finland, 2012.

\end{thebibliography}

%


\vspace{1cm}
\section*{Open-source software}
\vspace{-1cm}

\begin{thebibliography}{6}
\bibitem{runko}
    \textbf{Runko}, 
\newblock Modern \textsc{C++}14/\textsc{python3} toolbox for kinetic plasma simulations. 
 \newblock \mbox{\href{https://github.com/natj/Runko}{\nolinkurl{https://github.com/natj/runko}}}

\bibitem{corgi}
    \textbf{CORGI}, 
 \newblock \textsc{C++}14 grid infrastructure for massively parallel multi-physics simulations. 
 \newblock \mbox{\href{https://github.com/natj/corgi}{\nolinkurl{https://github.com/natj/corgi}}} 

\bibitem{mpi4cpp}
    \textbf{mpi4cpp}, 
\newblock User-friendly \textsc{C++}14 MPI headers with template metaprogramming. 
 \newblock \mbox{\href{https://github.com/natj/mpi4cpp}{\nolinkurl{https://github.com/natj/mpi4cpp}}}

\bibitem{bender}
    \textbf{Bender, ray tracing code}, 
        \newblock General relativistic ray tracing code for computing radiation from rapidly rotating oblate neutron stars in \textsc{Julia}/\textsc{python3}. 
 \newblock \mbox{\href{https://github.com/natj/bender}{\nolinkurl{https://github.com/natj/bender}}}

\bibitem{hydro}
    \textbf{Hydro, modular 2D hydrodynamical code} 
\newblock with unsplitted HLLC Rieman solver, second order Runge-Kutta time-stepping, and linear piecewise reconstruction written in pure \textsc{Julia}.
 \newblock \mbox{\href{https://github.com/natj/hydro}{\nolinkurl{https://github.com/natj/hydro}}}

\bibitem{cellularautomata}
    \textbf{CellularAutomata.jl}, 
\newblock \textsc{Julia} library for 1/2D elementary and totalistic Cellular automata modeling. 
\newblock \mbox{\href{https://github.com/natj/CellularAutomata.jl}{\nolinkurl{https://github.com/natj/CellularAutomata.jl}}}
\end{thebibliography}

\noindent
$+$ Smaller libraries and software available at \mbox{\href{https://github.com/natj}{\nolinkurl{https://github.com/natj}}}.






\end{document}

