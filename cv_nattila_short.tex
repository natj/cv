\documentclass[10pt]{article}
%\usepackage{array, xcolor, bibentry}
\usepackage{array, xcolor}
\usepackage[margin=1.8cm]{geometry}
\usepackage[colorlinks = true,
            linkcolor = blue,
            urlcolor  = blue,
            citecolor = blue,
            anchorcolor = blue]{hyperref}

\newcommand\blfootnote[1]{%
  \begingroup
  \renewcommand\thefootnote{}\footnote{#1}%
  \addtocounter{footnote}{-1}%
  \endgroup
}


% page headers
%\usepackage{fancyhdr}
%\pagestyle{fancy}
%\fancyhf{}
%\definecolor{gray75}{gray}{0.8}
%\cfoot{\footnotesize \textcolor{gray}{\thepage~/~5}}
%\cfoot{\footnotesize \thepage~/~5}


\title{\vspace{-35pt}\bfseries\Huge Joonas N\"attil\"a}% \\[15pt]
%\Huge Curriculum Vitae}
%\author{joonas.nattila@su.se}
\author{jan2174@columbia.edu}
\date{}
 
\newcolumntype{L}{>{\raggedleft}p{0.10\textwidth}}
\newcolumntype{R}{p{0.84\textwidth}}
\definecolor{lightgray}{gray}{0.8}
\newcommand\VRule{\color{lightgray}\vrule width 0.5pt}

%%%%%%%%%
\usepackage{etaremune}
\makeatletter
\long\def\thebibliography#1{%
  \section*{\refname}%
  \@mkboth{\MakeUppercase\refname}{\MakeUppercase\refname}
  \settowidth{\dimen0}{\@biblabel{#1}}%
  \setlength{\dimen2}{\dimen0}%
  \addtolength{\dimen2}{\labelsep}
  \sloppy
  \clubpenalty 4000 
  \@clubpenalty 
  \clubpenalty 
  \widowpenalty 4000
  \sfcode `\.\@m
  \renewcommand{\labelenumi}{\@biblabel{\theenumi}} % labels like [3], [2], [1]
  \begin{etaremune}[labelwidth=\dimen0,leftmargin=\dimen2]\@openbib@code
}
\def\endthebibliography{\end{etaremune}}
\def\@bibitem#1{%
  \item \if@filesw\immediate\write\@auxout{\string\bibcite{#1}{\the\value{enumi}}}\fi\ignorespaces
}
\makeatother
%%%%%%%%%%%%%%%%%%%%%

 
\begin{document}
%\pagenumbering{gobble}
\maketitle

\vspace{-35pt}
%\begin{minipage}[ht]{\textwidth}
%
\begin{minipage}[ht]{0.650\textwidth}
Sex: Male\\
Born: June 25th, 1989, Tornio, Finland\\
Nationality: Finnish citizen\\
Languages: Finnish (native), English, Swedish  \\
%\phantom{Languages:} 
\end{minipage}
%
\begin{minipage}[ht]{0.24\textwidth}
Department of Physics \\
Columbia University \\
550 W 120th Street\\
New York, NY-10027 \\
Tel: $+$1 646 229 3667 \\
\url{http://natj.github.io}
\end{minipage}
%\vspace{10pt}
%\end{minipage} 

\vspace{-15pt}
\section*{Research interests\blfootnote{\today}} 
\vspace{-5pt}

\noindent \textit{High-energy astrophysics}: \small{accretion (accretion disks); compact objects (neutron stars, black holes)}

\noindent \textit{Plasma physics}:           \small{collisionless plasma dynamics; turbulence; particle acceleration}

%\noindent \textit{Nuclear physics}:          \small{equation of state of cold ultra-dense matter}

%\noindent \textit{General relativity}:       \small{ray tracing}

\noindent \textit{Computer sciences}:        \small{high-performance computing; parallelization paradigms; machine learning; Julia language}

\noindent \textit{Statistics}:               \small{Bayesian inference; Monte Carlo methods}

\noindent \textit{Mathematics}:              \small{cellular automata models }

 

% Employment
%\vspace{-5pt}
\section*{Employment}
%\vspace{-3pt}
\begin{tabular}{L!{\VRule}R}
2021--2023 & {\bf Flatiron Research Fellow}, Flatiron Institute's Center for Computational Astrophysics, New York, USA. \\[0ex]
2019--2021 & {\bf Postdoctoral Research Scientist}, Columbia University, New York, USA. \\[0ex]
2018--2019 & {\bf Nordita Fellow}, Nordita (Nordic Institute for Theoretical Physics), Stockholm, Sweden. \\[0ex]
%2016     & {\bf Nordita Visiting Ph.D. Fellow}, Nordita, Stockholm, Sweden. \\ [1ex]
%         & \small{Constraining neutron star mass and radius.} \\[1ex]
%Summer 2013 & {\bf Research Assistant}, University of Oulu, Finland. \\
%         & \small{Constraining neutron star mass and radius.} \\[0.5ex]
%
%Summer 2012 & {\bf University Trainee}, University of Oulu, Finland. \\
%         & \small{Dependence of X-ray burst spectral evolution on accretion rate.} \\[0.5ex]
%
%Summer 2011 & {\bf Research Assistant}, University of Oulu, Finland. \\
%         & \small{Thermonuclear type-I X-ray bursts from neutron stars.} \\[0.5ex]

\end{tabular}

%\vspace{-5pt}
\section*{Education}
%\vspace{-3pt}

\begin{tabular}{L!{\VRule}R}
  2014--2017          & {\bf Ph.D. in Astrophysics (with honours)}, University of Turku, Finland.\\
  & \small{Supervisor: Prof. Juri Poutanen, Director of Tuorla Observatory.} \\
  & \footnotesize{Title: X-ray bursts as a tool to constrain the equation of state of the ultra-dense matter inside neutron stars} \\[1ex]
  
%  & \small{weighted GPA of $4.48/5$} / 4.5/10 GPA\\[1ex]  
    2012--2013          & {\bf M.Sc. in Astronomy}, University of Oulu, Finland. \\ % \scriptsize{($4.71/5$ GPA)}\\
%  & \small{weighted GPA of $4.75/5$} \\[1ex]  
  2008--2012          & {\bf B.Sc. in Physics}, University of Oulu, Finland. \\ %\scriptsize{($4.51/5$ GPA)}\\
%  & \small{weighted GPA of $4.48/5$} \\[1ex]   
\end{tabular}

%\vspace{-5pt}
\section*{Awards \& Recognitions}
%\vspace{-3pt}
\begin{tabular}{L!{\VRule}R}
  2019 & Joint Princeton/Flatiron Postdoctoral Research Fellowship, Princeton University (\textit{Declined}) \\
  2018 & Turku Finnish University Society Prize for best doctoral dissertation \\
  2018 & V\"ais\"al\"a Prize 2018: Prize for outstanding thesis in Astronomy \\
  2018 & PCS Best Doctoral Thesis of 2017 Prize \\
  2016 & Nordita Visiting Ph.D. Fellow \\
\end{tabular}

%\vspace{-5pt}
\section*{Teaching}
%\vspace{-3pt}
\begin{tabular}{L!{\VRule}R}
2019        & {\bf Visiting Lecturer, Computational fluid dynamics course}, Columbia University, USA. \\
            & \small{Visiting lecturer on ''Collisionless plasma simulations''.} \\[1ex]

2015---2019 & {\bf Lecturer, High Performance Computing Summer School}, CSC, Finland. \\
    \footnotesize{(5 times)} & \small{Lecturer \& tutor for Finnish IT Center for Science HPC Summer School.} \\[1ex]

2018, 2019  & {\bf Lecturer, Introduction to Julia}, CSC, Finland. \\
    \footnotesize{(2 times)} & \small{Lecturer for an introductory course on the Julia programming language.} \\[1ex]

2015---2017 & {\bf Lecturer, Software tools in Physics}, University of Turku, Finland. \\
    \footnotesize{(3 times)}  & \small{Lecturer of the ``Introduction to Unix'' section of the course (3 ECTS).} \\[1ex]
\end{tabular} 

\noindent
In addition, teaching assistant in \textbf{Optics} (2016; 6 ECTS) in Univ. Turku, and \textbf{Thermophysics} (3 times, 2011---2013; 6 ECTS), \textbf{Electricity and Magnetism} (2012; 4 ECTS), \textbf{Laboratory Exercises in Physics 1} (2 times, 2011---2012; 3 ECTS), \textbf{Mathematics of Physics} (2011; 6 ECTS), and \textbf{Waveforms and Optics} (2 times, 2011; 6 ECTS) in Univ. Oulu.


%--------------------------------------------------
%\newpage

%\vspace{+10pt}
\section*{Mentoring \& Supervision}
%\vspace{-5pt}
%Co-supervised 2 M.Sc. thesis, and 1 B.Sc thesis. 
%Currently co-supervising 1 PhD thesis.

\noindent
\begin{tabular}{L!{\VRule}R}
  2017--2020 & \textbf{Tuomo Salmi}, PhD. student, University of Turku, Finland. \\
  & \small{PhD. co-supervisor: Neutron star mass and radius constraints from pulse profile modeling.} \\[1ex]
  2019 \phantom{3000} & \textbf{John Hope}, MSc. student, University of Bath, UK. \\
  & \small{M.Sc. thesis supervisor: PIC simulations of relativistic collisionless shocks across different magnetizations.} \\[1ex]
  2015--2017 & \textbf{Jere Kuuttila}, co-supervisor for M.Sc. thesis, University of Turku, Finland. \\
%  & \small{M.Sc. co-supervisor} \\[0ex]
  2015--2016 & \textbf{Tuomo Salmi}, co-supervisor for M.Sc. thesis, University of Turku, Finland. \\
%  & \small{M.Sc. co-supervisor} \\[0ex]
  2014--2015 & \textbf{Jere Kuuttila}, co-supervisor for B.Sc. thesis, University of Turku, Finland. \\
%  & \small{B.Sc. co-supervisor} \\[0ex]
\end{tabular}


\vspace{5pt}
\noindent
%In addition, co-supervised \textbf{J. Kuuttila} (\textbf{M.Sc. thesis}; 2015--2017), \textbf{T. Salmi} (\textbf{M.Sc. thesis}; 2015--2016), and \textbf{J. Kuuttila} (\textbf{B.Sc. thesis}; 2014--2015).
In addition, Nordita host for visiting PhD. students:
\textbf{M. Bussov}, Dec. 2019;
\textbf{K. Smedt}, Nov. 2019; 
\textbf{T. Salmi}, May 2019.
\vspace{15pt}


%\vspace{-5pt}
%\section*{Professional duties}
%\vspace{-5pt}
%Referee for Astronomy \& Astrophysics, Monthly Royal Notices of Astronomy.
% Aggregator/Convenor for CompCoffee in Univ. Turku


\vspace{-5pt}
\section*{Presentations \& Talks\blfootnote{\today}}
\vspace{-3pt}
Most recent ones include:
\vspace{2pt}

\begin{tabular}{L!{\VRule}R}
  2020 & \textbf{\textit{Invited} IAS Coffee talk}, Princeton, USA. \\
  2020 & \textbf{\textit{Colloquium:} Kumpula Colloquium}, Helsinki, Finland. \\
  2020 & \textbf{\textit{Invited:} Nordita Dynamo Series}, Stockholm, Sweden. \\
  2020 & \textbf{Nordita Astro group meeting}, Stockholm, Sweden. \\
  2020 & \textbf{\textit{Invited:} St. Louis neutron star meeting}, St. Louis, USA. \\
  2020 & \textbf{CCA compact object meeting}, New York, USA. \\
  2019 & \textbf{Extreme Objects Meeting}, Stockholm, Sweden. \\
  2019 & \textbf{KITP conference on Multiscale modeling of Plasmas}, Santa Barbara, USA. \\
  2019 & \textbf{\textit{Invited:} Bursting the Bubble}, Lorentz center, Leiden, Netherlands. \\
  2019 & \textbf{\textit{Invited:} Astro Colloquium}, University of T\"ubingen, T\"ubingen, Germany. \\
%  2019 & \textbf{\textit{Invited:} Astro talk}, Uppsala University, Uppsala, Sweden. \\
%  2019 & \textbf{Extreme Objects Meeting}, Stockholm, Sweden. \\
%  2018 & \textbf{Astroplasmas: Particle acceleration and transport}, Rende, Italy. \\
%  2018 & \textbf{Tuorla-Tarto meeting}, Turku, Finland. \\
%  2018 & \textbf{\textit{Invited:}} \textbf{Time for Accretion}, Sigtuna, Sweden. \\
%  2018 & \textbf{Astroplasmas seminar}, Princeton, USA. \\
%  2018 & \textbf{High energy astro group meeting}, Columbia University, USA. \\
%  2018 & \textbf{Nordita Seminar}, Nordita, Sweden. \\
%  2018 & \textbf{\textit{Invited:}} \textbf{Astronomers' days}, Kuusamo, Finland. \\
%  2018 & \textbf{\textit{Invited:}} \textbf{Fire and Ice: Hot QCD meets cold and dense matter}, Saariselk\"a, Finland. \\
%  2017 & \textbf{\textit{Invited:}} \textbf{Holographic dense QCD and neutron stars}, ENS, Paris, France. \\
% 2017 & \textbf{Astrophysics Seminar}, Helsinki, Finland. \\
% 2017 & \textbf{Exascale thinking of particle energization problems}, Nordita, Sweden. \\
% 2016 & \textbf{From quarks to gravitational waves: Neutron stars as a laboratory for fundamental physic}, CERN. \\
% 2016 & \textbf{COSPAR 2016, E1.1: Accreting Neutron Stars and Stellar-mass Black Hole}, Istanbul, Turkey. (\textit{Conference canceled!})\\
% 2016 & \textbf{INT-16-2b: Phases Of Dense Matter Workshop}, Seattle, USA. \\
% 2016 & \textbf{JINA-CEE Symposium: Neutron Stars in the Multi-Messenger Era}, Ohio, USA. \\
% 2016 & \textbf{Nordita Workshop on accretion onto magnetized neutron stars}, Stockholm, Sweden. \\
% 2015 & \textbf{Workshop on Relativistic Astrophysics}, Kavalto, Finland. \\
% 2015 & \textbf{University of Maryland, Colloquium speaker}, Washington, USA.\\
% 2015 & \textbf{University of Tennessee, Colloquium speaker}, Tennessee, USA.\\
% 2015 & \textbf{The Neutron Star Radius, And All That Jazz}, Montreal, Canada.\\
% 2015 & \textbf{40 years of X-ray bursts: Extreme explosions in dense environments}, Madrid, Spain.\\
% 2014 & \textbf{ESAC (visiting scientist presentation)}, Madrid, Spain. \\
% 2014 & \textbf{Physics of Neutron Stars Conference}, St. Petersburg, Russia. \\
% 2014 & \textbf{Astronomers' Days}, Savonlinna, Finland. \\
% 2013 & \textbf{European Week of Astronomy and Space Science}, Turku, Finland. \\
% 2012 & \textbf{Astronomers' Days}, Porvoo, Finland. \\
\end{tabular}

\noindent
In total 1 colloquium, 13 invited, 24 contributed talks.

%--------------------------------------------------

\vspace{-5pt}
\section*{Funding}
%\section*{Research \& Travel funding}
%\vspace{-1.1cm}
\subsection*{\phantom{sub} Research}
\begin{tabular}{L!{\VRule}R}
    2020  & $\mathbf{\sim60\,000~\mathbf{eur~}}$ \textbf{Wenner-Gren grant Co-I}: Postdoc grant for Surajit Mondal \\
          & Reconnection, Radio observations, and switchbacks. 
    \\[0.5ex]
%  2015--2017 & $\mathbf{\sim82\,000~\mathbf{eur}~}$ \textbf{UTUGS Physical and Chemical Sciences funded Ph.D. position}\\
    2016  & $\mathbf{\sim2\,000~\mathbf{eur}~}$ \textbf{Magnus Ehrnrooth Foundation} Travel grant\\[0.5ex]
  2015--2017 & \textbf{UTUGS Physical and Chemical Sciences funded 3yr. Ph.D. scholarship}\\
    & \small{Constraining neutron star mass and radius.}\\[0.5ex]
  2015--2016 & $\mathbf{23\,000~\mathbf{eur}~}$ \textbf{V\"ais\"al\"a Foundation grant} \\
    & \small{Magnetar atmosphere models} (\textit{Declined}) \\[0.5ex]
  2014--2015 & $\mathbf{23\,000~\mathbf{eur}~}$ \textbf{V\"ais\"al\"a Foundation grant} \\
    & \small{Magnetar atmosphere models: breaking the barrier between observations and theory} \\[0.5ex]
\end{tabular}

$+$ Some smaller travel grants (in total $\sim 10$k eur).
 
%\vspace{-15pt}
%\subsection*{\phantom{sub} Travel}
%%\subsubsection*{Travel funding}
%\begin{tabular}{L!{\VRule}R}
%  2016 & $ \mathbf{\sim1\,000~\mathbf{eur}~}$ \small{\textbf{CERN} From quarks to gravitational waves workshop.}\\
%  2016 & $ \mathbf{\sim2\,000~\mathbf{eur}~}$ \small{\textbf{Magnus Ehrnrooth Foundation} JINA-CEE symposium (Ohio) and COSPAR 2016 (Istanbul).}\\
%  2016 & $ \mathbf{\sim1\,000~\mathbf{eur}~}$ \small{\textbf{UTUGS Physical and Chemical Sciences} JINA-CEE symposium (Ohio).}\\
%  2016 & $ \mathbf{\sim2\,000~\mathbf{eur}~}$ \small{\textbf{ESAC} Visiting scientist (host: Jari Kajava). }\\
%%  2015 & $ \mathbf{\sim500~\mathbf{eur}~}   $ \small{\textbf{University of Tennessee} Research visit (host: Andrew Steiner).}\\
%%  2015 & $ \mathbf{\sim500~\mathbf{eur}~}   $ \small{\textbf{University of Maryland} Research visit (host: Cole Miller).}\\
%  2015 & $ \mathbf{\sim1\,000~\mathbf{eur}~}$ \small{\textbf{UTUGS Physical and Chemical Sciences} The Neutron Star Radius, and All That Jazz -conference, Montreal.}\\
%  2015 & $ \mathbf{\sim1\,000~\mathbf{eur}~}$ \small{\textbf{ESAC} 40 years of X-ray bursts - conference.}\\
%  2014 & $ \mathbf{\sim1\,000~\mathbf{eur}~}$ \small{\textbf{ESAC} Research visit (host: Jari Kajava).}\\
%\end{tabular}
%
%$+$ Some smaller travel grants.

%\vspace{-8pt}
%\subsection*{\phantom{sub} Observation time}
%\begin{tabular}{L!{\VRule}R}
%    2018 & \textbf{NuSTAR/INTEGRAL/XMM-Newton ToO time (30ks/170ks/100ks)} \\
%         &  \small{Co-I, Proposal 1540022:} \footnotesize{Measuring the High Energy Emission of Millisecond X-Ray Pulsars in Outburst} \\[1ex]
%\end{tabular}

\vspace{-8pt}
\subsection*{\phantom{sub} Supercomputer time}
\begin{tabular}{L!{\VRule}R}
    2021 & $ \mathbf{\sim22\mathbf{M}~\mathbf{CPU h}~}$ \textbf{SNIC/Beskow}, \small{\textbf{Co-PI}: Astrophysical turbulence and dynamo action} \\
    2020 & $ \mathbf{\sim22\mathbf{M}~\mathbf{CPU h}~}$ \textbf{SNIC/Beskow/Kebnekaise}, \small{\textbf{Co-PI}: Astrophysical turbulence and dynamo action} \\
    2019 & $ \mathbf{\sim22\mathbf{M}~\mathbf{CPU h}~}$ \textbf{SNIC/Beskow/Kebnekaise}, \small{\textbf{Co-PI}: Astrophysical turbulence and dynamo action} \\
    2018 & $ \mathbf{\sim60\mathbf{k}~\mathbf{CPU h}~}$ \textbf{SNIC/Kebnekaise}, \small{\textbf{PI}: Relativistic plasma in silico (testing of \textsc{Runko}).} \\
\end{tabular}

\vspace{-5pt}
\section*{Professional Societies and Services}
\vspace{-5pt}
\begin{tabular}{L!{\VRule}R}
    2019--\phantom{3000} & Member of Young Academy Finland (under Finnish Academy of Sciences and Letters) \\
    2018--\phantom{3000} & IAU Junior member \\
    2017\phantom{--3000} & Organizer \& Convener for CompCoffee meetings \small{(weekly meetings to discuss computational problems)} \\
    2016--\phantom{3000} & eXTP Dense Matter science working group \\
%    2016--\phantom{3} & Referee for Monthly Notices of Royal Astronomical Society, Astronomy \& Astrophysics \\
    2015--2018        & ESA XIPE satellite Science Team (SWG2.2 Accreting Millisecond Pulsars)  \\
    2014--\phantom{3000} & Member of organizing committee for CSC HPC Summer Schools \\
    2013--2019 & Member of JuliaLang organization \small{(Open source community for Julia programming language)}\\
    2012--\phantom{3000} & Member of Finnish Astronomical Society \\[2ex]
\end{tabular}

\noindent 
In addition, referee for 
ApJL,
MNRAS, 
A\&A, 
ApJ, 
Phys. Rev. D.,
European Physical Journal A, 
and Universe.
\vspace{5pt}



\vspace{-5pt}
\section*{Conference organization}
\vspace{-5pt}
\begin{tabular}{L!{\VRule}R}
  2017     & \textbf{Nordita Workshop: Exascale thinking of particle energization problems}, Stockholm, Sweden. \\
  & Member of the scientific and local organizing committee. \\[1ex]

  2015     & \textbf{Workshop on Relativistic Astrophysics}, Kavalto, Finland. \\
  & Member of the local organizing committee. \\[1ex]

  2015     & \textbf{PCS Annual Seminar day}, University of Turku, Finland. \\
  & Chairman \& member of the organizing committee.\\[1ex]
  
\end{tabular}


\vspace{-5pt}
\section*{Public outreach}
\vspace{-5pt}
My research has been presented in various media: 
\vspace{3pt}

\begin{tabular}{L!{\VRule}R}
    2020 & \textbf{Published text on Finnish science magazine Q\&A section} \\
         & "How can neutron stars have a magnetic field?" T\"ahdet \& Avaruus Sep. 2020 issue.\\[1ex]

    2020 & \textbf{Published text on Finnish science magazine Q\&A section} \\
         & "What happens to neutron star matter outside the star?" T\"ahdet \& Avaruus Oct. 2020 issue.\\[1ex]

    2020 & \textbf{Meet the scientist}: Educational science video on "Gravitation". \\
         & Popular science video targeted for high school students, \href{https://www.youtube.com/watch?v=Ch38VpF341I}{youtube.com/watch?v=Ch38VpF341I} \\[1ex]

    2019 & \textbf{Academy Club for Young Scientist}: Public science talk. \\
         & Astrophysical turbulence: from stirring coffee to mixing galaxies; \href{https://www.youtube.com/watch?v=W7ljVlSEAX4}{youtube.com/watch?v=W7ljVlSEAX4} \\[1ex]

    2019---2020 & \textbf{On the possibility of quark matter cores in neutron star} (Annala et al. 2020)\\
         & Incl.:
         \href{https://astrobites.org/2019/03/29/a-strange-type-of-matter-may-lie-at-the-heart-of-neutron-stars/}{Astrobites},
         \href{https://www.universetoday.com/146476/neutron-stars-could-have-a-layer-of-exotic-quark-matter-inside-them/}{Universe Today},
         \href{https://physicsworld.com/a/neutron-stars-may-contain-free-quarks/}{Physics World},
         \href{https://en.wikipedia.org/wiki/QCD_matter}{Wikipedia on QCD matter},
         T\"ahdet \& Avaruus 4/2019 \\[1ex]

    2019 & \textbf{Twitter AMA on scientists abroad}  \\
         & In part of \href{https://www.timeoutdialogue.fi/}{TimeoutDialogue/Er\"atauko society}.\\[1ex]

    2018 & \textbf{Personal profile} on T\"ahdet \& Avaruus Finnish science magazine.\\
         & T\"ahdet \& Avaruus Feb 2018 issue.\\[1ex]

    2017 & \textbf{Groundbreaking new neutron star radius measurement} (N\"attil\"a et al. 2017) \\
         & Incl.:
  \href{https://cosmosmagazine.com/space/nuke-blasts-reveal-true-size-of-neutron-stars}{Cosmos 27.11.2017},
    \href{https://phys.org/news/2017-11-method-neutron-star-size-based.html}{Phys.org}.
  \href{https://www.avaruus.fi/uutiset/tahdet-sumut-ja-galaksit/turkulaiset-keksivat-uuden-tavan-mitata-neutronitahtien-kokoa.html}{T{\"a}hdet \& Avaruus (25.11.2017)},
  \href{https://www.turkulainen.fi/artikkeli/578926-turun-yliopiston-tutkimusryhma-kehitti-tavan-mitata-neutronitahtien-kokoa}{Turkulainen (10.11.2017)},
  \href{http://www.ts.fi/uutiset/paikalliset/3724265/uusi+menetelma+mahdollistaa+neutronitahtien+sateen+mittauksen+galaksin+toiselta+laidalta}{Turun Sanomat (10.11.2017)},
  \href{http://www.aamuset.fi/uutiset/3758822/kosmiset+ydinrajahdykset+tuovat+uutta+tietoa+neutronitahtien+rakenteesta}{Aamuset (8.12.2017)},
  \href{https://www.tekniikkatalous.fi/tiede/avaruus/neutronitahtien-tutkija-kaytti-apunaan-nasa-n-satelliitteja-kynaa-ja-paperia-kuutiosentti-neutronimateriaa-painaa-uskomattomat-100-miljoonaa-tonnia-6691137}{Tekniikka \& Talous (8.12.2017)},
  \href{https://www.verkkouutiset.fi/kosmisista-ydinrajahdyksista-uutta-tietoa-neutronitahtien-rakenteesta/}{Verkkouutiset (8.12.2017)} \\[3ex]

  2016 & \textbf{Detection of burning ashes from neutron stars} (Kajava, N\"attil\"a, et al 2017) \\
         & \href{http://www.tiedetuubi.fi/avaruus/suomalaistutkijat-varmistivat-uuden-tavan-tehda-alkuaineita-loyto-voi-auttaa-selvittamaan}{Tiedetuubi.fi (30.11.2016)} \\[3ex]
\end{tabular}

%My research has been presented in various local (Finnish) media: 
%\begin{tabular}{L!{\VRule}R}
%  2019 & T\"ahdet ja Avaruus (4/2019) \\
%  2017 & \href{https://www.avaruus.fi/uutiset/tahdet-sumut-ja-galaksit/turkulaiset-keksivat-uuden-tavan-mitata-neutronitahtien-kokoa.html}{t{\"a}hdet \& avaruus (25.11.2017)} \\
%  2017 & \href{https://www.turkulainen.fi/artikkeli/578926-turun-yliopiston-tutkimusryhma-kehitti-tavan-mitata-neutronitahtien-kokoa}{turkulainen (10.11.2017)} \\
%  2017 & \href{http://www.ts.fi/uutiset/paikalliset/3724265/uusi+menetelma+mahdollistaa+neutronitahtien+sateen+mittauksen+galaksin+toiselta+laidalta}{turun sanomat (10.11.2017)} \\
%  2017 & \href{http://www.aamuset.fi/uutiset/3758822/kosmiset+ydinrajahdykset+tuovat+uutta+tietoa+neutronitahtien+rakenteesta}{aamuset (8.12.2017)} \\
%  2017 & \href{https://www.tekniikkatalous.fi/tiede/avaruus/neutronitahtien-tutkija-kaytti-apunaan-nasa-n-satelliitteja-kynaa-ja-paperia-kuutiosentti-neutronimateriaa-painaa-uskomattomat-100-miljoonaa-tonnia-6691137}{tekniikka \& talous (8.12.2017)} \\
%  2017 & \href{https://www.verkkouutiset.fi/kosmisista-ydinrajahdyksista-uutta-tietoa-neutronitahtien-rakenteesta/}{verkkouutiset (8.12.2017)} \\
%  2016 & \href{http://www.tiedetuubi.fi/avaruus/suomalaistutkijat-varmistivat-uuden-tavan-tehda-alkuaineita-loyto-voi-auttaa-selvittamaan}{tiedetuubi.fi (30.11.2016)} \\
%\end{tabular}

%TT lehti x2
%\href{http://www.tiedetuubi.fi/avaruus/suomalaistutkijat-varmistivat-uuden-tavan-tehda-alkuaineita-loyto-voi-auttaa-selvittamaan}{tiedetuubi.fi (30.11.2016)},
%\href{http://www.ts.fi/uutiset/paikalliset/3724265/Uusi+menetelma+mahdollistaa+neutronitahtien+sateen+mittauksen+galaksin+toiselta+laidalta}{Turun Sanomat (10.11.2017)},
%\href{https://www.turkulainen.fi/artikkeli/578926-turun-yliopiston-tutkimusryhma-kehitti-tavan-mitata-neutronitahtien-kokoa}{Turkulainen (10.11.2017)},
%\href{https://www.avaruus.fi/uutiset/tahdet-sumut-ja-galaksit/turkulaiset-keksivat-uuden-tavan-mitata-neutronitahtien-kokoa.html}{T{\"a}hdet \& Avaruus (25.11.2017)},
%\href{http://www.aamuset.fi/uutiset/3758822/Kosmiset+ydinrajahdykset+tuovat+uutta+tietoa+neutronitahtien+rakenteesta}{Aamuset (8.12.2017)},
%\href{https://www.tekniikkatalous.fi/tiede/avaruus/neutronitahtien-tutkija-kaytti-apunaan-nasa-n-satelliitteja-kynaa-ja-paperia-kuutiosentti-neutronimateriaa-painaa-uskomattomat-100-miljoonaa-tonnia-6691137}{Tekniikka \& Talous (8.12.2017)},
%\href{https://www.verkkouutiset.fi/kosmisista-ydinrajahdyksista-uutta-tietoa-neutronitahtien-rakenteesta/}{Verkkouutiset (8.12.2017)}, 
%T\"ahdet ja Avaruus (4/2019);

%\noindent
%And in international media:
%
%\begin{tabular}{L!{\VRule}R}
%  2017 & \href{https://cosmosmagazine.com/space/nuke-blasts-reveal-true-size-of-neutron-stars}{Cosmos 27.11.2017} \\
%  2017 & \href{https://phys.org/news/2017-11-method-neutron-star-size-based.html}{Phys.org}
%\end{tabular}

%\href{https://cosmosmagazine.com/space/nuke-blasts-reveal-true-size-of-neutron-stars}{Cosmos 27.11.2017},
%\href{https://phys.org/news/2017-11-method-neutron-star-size-based.html}{Phys.org}.
%\vspace{15pt}

%\begin{tabular}{L!{\VRule}R}
%%  2016     & Finnish Astronomical magazine (\textit{T\"ahdet ja Avaruus}) showcased our work on X-ray bursts. \textit{In prep.} \\[1ex]
%  2016     & Finnish science blog reported on our work about heavy metal enrichment of the Universe from thermonuclear X-ray bursts  \\[1ex]
%\end{tabular}


%Little space between publication part
%\section*{}
%\section*{}
%\section*{}
%\setcounter{page}{0}
\newpage


%Ruma kun mikä, mutta toimii
%Jani Lappalainen 2011

\newcommand{\aj}{AJ}%
          % Astronomical Journal
\newcommand{\actaa}{Acta Astron.}%
          % Acta Astronomica
\newcommand{\araa}{ARA\&A}%
          % Annual Review of Astron and Astrophys
\newcommand{\apj}{ApJ}%
          % Astrophysical Journal
\newcommand{\apjl}{ApJ}%
          % Astrophysical Journal, Letters
\newcommand{\apjs}{ApJS}%
          % Astrophysical Journal, Supplement
\newcommand{\ao}{Appl.~Opt.}%
          % Applied Optics
\newcommand{\apss}{Ap\&SS}%
          % Astrophysics and Space Science
\newcommand{\aap}{A\&A}%
          % Astronomy and Astrophysics
\newcommand{\aapr}{A\&A~Rev.}%
          % Astronomy and Astrophysics Reviews
\newcommand{\aaps}{A\&AS}%
          % Astronomy and Astrophysics, Supplement
\newcommand{\azh}{AZh}%
          % Astronomicheskii Zhurnal
\newcommand{\baas}{BAAS}%
          % Bulletin of the AAS
\newcommand{\bac}{Bull. astr. Inst. Czechosl.}%
          % Bulletin of the Astronomical Institutes of Czechoslovakia 
\newcommand{\caa}{Chinese Astron. Astrophys.}%
          % Chinese Astronomy and Astrophysics
\newcommand{\cjaa}{Chinese J. Astron. Astrophys.}%
          % Chinese Journal of Astronomy and Astrophysics
\newcommand{\icarus}{Icarus}%
          % Icarus
\newcommand{\jcap}{J. Cosmology Astropart. Phys.}%
          % Journal of Cosmology and Astroparticle Physics
\newcommand{\jrasc}{JRASC}%
          % Journal of the RAS of Canada
\newcommand{\mnras}{MNRAS}%
          % Monthly Notices of the RAS
\newcommand{\memras}{MmRAS}%
          % Memoirs of the RAS
\newcommand{\na}{New A}%
          % New Astronomy
\newcommand{\nar}{New A Rev.}%
          % New Astronomy Review
\newcommand{\pasa}{PASA}%
          % Publications of the Astron. Soc. of Australia
\newcommand{\pra}{Phys.~Rev.~A}%
          % Physical Review A: General Physics
\newcommand{\prb}{Phys.~Rev.~B}%
          % Physical Review B: Solid State
\newcommand{\prc}{Phys.~Rev.~C}%
          % Physical Review C
\newcommand{\prd}{Phys.~Rev.~D}%
          % Physical Review D
\newcommand{\pre}{Phys.~Rev.~E}%
          % Physical Review E
\newcommand{\prl}{Phys.~Rev.~Lett.}%
          % Physical Review Letters
\newcommand{\pasp}{PASP}%
          % Publications of the ASP
\newcommand{\pasj}{PASJ}%
          % Publications of the ASJ
\newcommand{\qjras}{QJRAS}%
          % Quarterly Journal of the RAS
\newcommand{\rmxaa}{Rev. Mexicana Astron. Astrofis.}%
          % Revista Mexicana de Astronomia y Astrofisica
\newcommand{\skytel}{S\&T}%
          % Sky and Telescope
\newcommand{\solphys}{Sol.~Phys.}%
          % Solar Physics
\newcommand{\sovast}{Soviet~Ast.}%
          % Soviet Astronomy
\newcommand{\ssr}{Space~Sci.~Rev.}%
          % Space Science Reviews
\newcommand{\zap}{ZAp}%
          % Zeitschrift fuer Astrophysik
\newcommand{\nat}{Nature}%
          % Nature
\newcommand{\iaucirc}{IAU~Circ.}%
          % IAU Cirulars
\newcommand{\aplett}{Astrophys.~Lett.}%
          % Astrophysics Letters
\newcommand{\apspr}{Astrophys.~Space~Phys.~Res.}%
          % Astrophysics Space Physics Research
\newcommand{\bain}{Bull.~Astron.~Inst.~Netherlands}%
          % Bulletin Astronomical Institute of the Netherlands
\newcommand{\fcp}{Fund.~Cosmic~Phys.}%
          % Fundamental Cosmic Physics
\newcommand{\gca}{Geochim.~Cosmochim.~Acta}%
          % Geochimica Cosmochimica Acta
\newcommand{\grl}{Geophys.~Res.~Lett.}%
          % Geophysics Research Letters
\newcommand{\jcp}{J.~Chem.~Phys.}%
          % Journal of Chemical Physics
\newcommand{\jgr}{J.~Geophys.~Res.}%
          % Journal of Geophysics Research
\newcommand{\jqsrt}{J.~Quant.~Spec.~Radiat.~Transf.}%
          % Journal of Quantitiative Spectroscopy and Radiative Trasfer
\newcommand{\memsai}{Mem.~Soc.~Astron.~Italiana}%
          % Mem. Societa Astronomica Italiana
\newcommand{\nphysa}{Nucl.~Phys.~A}%
          % Nuclear Physics A
\newcommand{\physrep}{Phys.~Rep.}%
          % Physics Reports
\newcommand{\physscr}{Phys.~Scr}%
          % Physica Scripta
\newcommand{\planss}{Planet.~Space~Sci.}%
          % Planetary Space Science
\newcommand{\procspie}{Proc.~SPIE}%
          % Proceedings of the SPIE
\section*{Publications --- Joonas N\"attil\"a}
25 refereed publications; 
incl. 
\textit{Nature Physics} (1), 
\textit{PRL} (1),
\textit{ApJ} (4),
\textit{A\&A} (11),
\textit{MNRAS} (6).

\noindent
In total 629 citations since 2014; h-index 12, g-index 24, i10-index 13 (\href{http://adsabs.harvard.edu/cgi-bin/abs_connect?author=nattila,+J.&aut_syn=YES&return_req=no_params}{\nolinkurl{ADS}}).

%--------------------------------------------------
\subsection*{\phantom{sub} Peer-reviewed scientific articles}

\vspace{-20pt}
\renewcommand\refname{\phantom{bla}}
\bibliographystyle{hunsrt}
\bibliography{cv_pubs}

\nocite{*}



%--------------------------------------------------
\subsection*{\phantom{sub} Proceedings}
\vspace{-20pt}
\begin{thebibliography}{1}
\vspace{-5pt}

\bibitem{INTEGRAL}
E. {Annala}, T. {Gorda}, A. {Kurkela}, \textbf{J. {N{\"a}ttil{\"a}}}, and A. {Vuorinen}.
\newblock {Constraining the properties of neutron-star matter with observations}.
\newblock In {\em 12th INTEGRAL Conference}, Geneva, Switzerland, 11-15 February 2019
[\href{https://arxiv.org/abs/1904.01354}{\nolinkurl{arXiv:1904.01354}}].

\bibitem{XIPE}
P.~{Soffitta}, R.~{Bellazzini}, E.~{Bozzo}, V.~{Burwitz}, A.~{Castro-Tirado},
  E.~{Costa}, T.~{Courvoisier}, H.~{Feng}, S.~{Gburek}, R.~{Goosmann}, and
  et~al. (incl. \textbf{J}. \textbf{{N{\"a}ttil{\"a}}} )
\newblock {XIPE: the x-ray imaging polarimetry explorer}.
\newblock In {\em Space Telescopes and Instrumentation 2016: Ultraviolet to
  Gamma Ray}, volume 9905 of {\em \procspie}, page 990515, July 2016.
\href{https://doi.org/10.1117/12.2233046}{\nolinkurl{doi.org/10.1117/12.2233046}}.

\end{thebibliography}

%\newpage
%--------------------------------------------------
%\subsection*{}
%\subsection*{}
\subsection*{\phantom{sub} Theses}

\vspace{-20pt}
%\bibliographystyle{hunsrt}
\begin{thebibliography}{3}
\vspace{-5pt}

\bibitem{NatjThesis}
\textbf{J. N\"attil\"a}.
\newblock{X-ray bursts as a tool to constrain the equation of state of the ultra-dense matter inside neutron stars}.
\newblock PhD thesis, University of Turku, Finland, 2017. \href{http://urn.fi/URN:ISBN:978-951-29-7057-5}{\nolinkurl{ISBN:978-951-29-7057-5}}.

\bibitem{NatjMaster}
\textbf{J. N\"attil\"a}.
\newblock {Mass and radius constraints for neutron stars using the cooling tail method}.
\newblock Master's thesis, University of Oulu, Finland, 2013. \href{http://urn.fi/URN:NBN:fi:oulu-201312041966}{\nolinkurl{oulu-201312041966}}.

\bibitem{NatjBachelor}
\textbf{J. N\"attil\"a}.
\newblock {Spectral analysis of X-ray bursts from neutron stars: IGR
  J1747--2721 (\textit{Neutronit\"ahtien r\"ontgenpurkaukset ja niiden
  spektrianalyysi: IGR J1747--2721})}.
\newblock Bachelor's thesis, University of Oulu, Finland, 2012.

\end{thebibliography}

\vspace{-5pt}
\subsection*{\phantom{sub} Open source software}

\vspace{-20pt}
\begin{thebibliography}{6}
\vspace{-5pt}
\bibitem{runko}
    \textbf{Runko}, 
\newblock Modern \textsc{C++}14/\textsc{python3} toolbox for kinetic plasma simulations. 
 \newblock \mbox{\href{https://github.com/natj/Runko}{\nolinkurl{https://github.com/natj/runko}}}

\bibitem{corgi}
    \textbf{CORGI}, 
 \newblock \textsc{C++}14 grid infrastructure for massively parallel multi-physics simulations. 
 \newblock \mbox{\href{https://github.com/natj/corgi}{\nolinkurl{https://github.com/natj/corgi}}} 

\bibitem{mpi4cpp}
    \textbf{mpi4cpp}, 
\newblock User-friendly \textsc{C++}14 MPI headers with template metaprogramming. 
 \newblock \mbox{\href{https://github.com/natj/mpi4cpp}{\nolinkurl{https://github.com/natj/mpi4cpp}}}

\bibitem{bender}
    \textbf{Bender, ray tracing code}, 
        \newblock General relativistic ray tracing code for computing radiation from rapidly rotating oblate neutron stars in \textsc{Julia}/\textsc{python3}. 
 \newblock \mbox{\href{https://github.com/natj/bender}{\nolinkurl{https://github.com/natj/bender}}}

\bibitem{hydro}
    \textbf{Hydro, modular 2D hydrodynamical code} 
\newblock with unsplitted HLLC Rieman solver, second order Runge-Kutta time-stepping, and linear piecewise reconstruction written in pure \textsc{Julia}.
 \newblock \mbox{\href{https://github.com/natj/hydro}{\nolinkurl{https://github.com/natj/hydro}}}

\bibitem{cellularautomata}
    \textbf{CellularAutomata.jl}, 
\newblock \textsc{Julia} library for 1/2D elementary and totalistic Cellular automata modeling. 
\newblock \mbox{\href{https://github.com/natj/CellularAutomata.jl}{\nolinkurl{https://github.com/natj/CellularAutomata.jl}}}
\end{thebibliography}

\noindent
$+$ Smaller libraries and software available at \mbox{\href{https://github.com/natj}{\nolinkurl{https://github.com/natj}}}.

\end{document}

